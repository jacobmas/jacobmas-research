\section{Improved Signature Scheme With Tight Security}
\label{sec:func-eval}

\jnote{TODO: 1) Separate out the W-PRF secret key (which should be
  $\vecs \in \calT^{\kappa-2}$ and the final 2 elements, which
  should be $z^{(0)}=0$, $z^{(1)}=1$.)
}
\jnote{TODO: 2) $\tdfencode$ may stretch the input, so $\calT^{\ell}$
  is what it should have as the output space}


In this section, we define our improved signature scheme, and prove
security with a tight reduction. We also discuss how to 
homomorphically evaluate our weak PRF in an efficient manner. 

To avoid the clunky notation that has characterized many lattice-based
constructions (and indeed, many cryptographic constructions in general), we write
the scheme generically in terms of the PHTDF cryptographic primitive,
as well as in terms of a generic weak PRF and chameleon hash function,
and then focus on the homomorphic evaluation details in Section~\ref{sec:weak-prf-evaluation}.

\subsection{Parameters}
\label{sec:sig-params}
Here we discuss the basic parameters of the scheme and the
requirements they need to satisfy. All parameters in the scheme are
parameterized by an underlying security $\lambda$. 
We have that $\alpha,w=\theta(\lambda)$. We will have that $\calT=\Z_q$.  The form of the
parameter $\kappa\geq 2$ depends on the concrete instantiation and evaluation of the
PHTDF. We defer details to
Section~\ref{sec:weak-prf-evaluation}. 

We need $\weakprf: \calT^{\kappa-2} \times \calT^{w-1} \to \bit$ be an
$(\epsilon_{\weakprf},t_{\weakprf})$-secure weak pseudorandom
function, where $\calT^{\kappa-2}$ is the keyspace and $\calT^{w-1}$ is the
input space. 

For convenience reasons, we also need the PRF to be trivial in the
(negligibly likely) case that $\calT^{k-2}=\veczero$; formally, we need $\weakprf(\veczero,\cdot)=0$ for all
$\vecy \in \calT^{w-1}$. This property is satisfied naturally by $\lwr$, and it
and we can construct such a weak PRF from an arbitrary weak PRF
$\weakprf'$ by setting

\[\weakprf(\vecs,\vecy):=\weakprf'(\vecs,\vecy)\oplus
\weakprf'(\veczero,\vecy).\]



We also need $\ch = (\chgen,\chhash,\chinv)$ to be a secure chameleon
hash function family such that the input space $\calX$ is efficiently
sampleable and grows in size as a
(polynomial) function inthe underlying security parameter $\lambda$
and such that the output space
$\calY=\calT^{w-1}$. This can be achieved essentially without loss of
generality, as long as there is an efficiently
computable embedding of the output space $\calY$ into
$\calT^{w-1}$. We use $w-1$ instead $w$ because we will be appending a
bit $b$ to $\vecy \in \calT^{w-1}$ as input to the function $g$ that
will be evaluated homorphically.\jnote{TODO: Cite specific instantiation}

We define $g:\calT^{\kappa} \times \calT^{w} \to \calT$ as
\begin{equation}\label{eq:def-g}
g((\vecs,z^{(0)},z^{(1)}),(\vecy,b))=(1-z^{(b)})-\weakprf(\vecs,\vecy).
\end{equation}

In the above equation, $\vecy \in \calT^{w-1}$ will be the output of a
chameleon hash function, and $\vecs \in \calT^{\kappa-2}$ will be the
secret key for the weak PRF. 



Finally, we need $\phtdf=(\phtdf.\tdfgen, \phtdf.\gentrap,$\\ 
$\phtdf.f,\phtdf.\tdfinv,\phtdf.\evaltd,\phtdf.\evalfunc)$ to be an $(\epsilon_{\phtdf},t_{\phtdf},g,\calS)$
CRP-secure PHTDF family, where $\calS=\calT^{\kappa-2}$. 



\subsection{Construction}
\label{sec:construction}
We explicitly sample the tag encoded in the
public key and add it to the signing key for use in
signing. We also make the use of a chameleon hash function explicit,
and we implicitly require the input space $\calX$ for the chameleon
hash function to satisfy shortness properties (and be such that $\abs{\calX}=2^{\Omega(\lambda)}$, which is indeed the
case for the Ducas-Micciancio chameleon hash function family construction.

\paragraph{Encoding the PRF secret key.}\label{sec:prf-sec-encoding}
Often, including in the instantiation we detail below in Section~\ref{sec:weak-prf-evaluation}, the
way of encoding the weak PRF secret key $\vecs$ into the verification key
will be something other than its ``natural representation,'' and we
abstract this out with the additional utility function $\tdfencode:\calT^{\kappa} \to \calT^{\ell}$.



\begin{description}
\item[$\siggen(1^{\lambda})$] On input security parameter $\lambda$,
\begin{enumerate}
\item Choose $v \gets \calV$, $\vecs\gets
  \calT^{\kappa-2}$. Set $z^{(0)}=0,z^{(1)}=1$. Compute $\tilde{\vecz}
  = \tdfencode(\vecs, z^{(0)}, z^{(1)}) \in \calT^{\ell}$.
\item Compute $(ek,td) \gets \ch.\chgen(1^{\lambda})$, $pk \gets \phtdf.\tdfgen(1^{\lambda})$
\item Compute  $\{(a_i, r_i) \gets \phtdf.\gentrap(pk,\tilde{\vecz}_i)\}_{i \in [\ell]}$.
\item Output $vk=(pk,\veca=\{a_i\}_{i \in [\ell]},ek,v)$, $sk=(\vecr=\{r_i\}_{i \in [\ell]},td,\vecz)$
\end{enumerate}
\item[$\sigsign(vk,sk,\mu)$] On input message $\mu \in \bit^{\alpha}$, secret key $sk$,
  verification key $vk$:
\begin{enumerate}
\item Sample the input randomness to the chameleon hash function $x
  \gets \calX$, compute the output $\vecy=\ch.\chhash(x,\mu)$, and
  then evaluate the weak pseudorandom function at $\vecy$ to get $b=\weakprf(\vecs,\vecy)$.
\item Perform the homomorphic evaluation of $g$ over the PHTDF to get $r^* \gets \evaltd(g, (\veca,\vecr),(\vecy,b))$, $a^* \gets \evalfunc(g,\veca,(\vecy,b)).$
\item 
Invert the trapdoor function to compute $u \gets \phtdf.\tdfinv_{r^*,pk,a^*,x,s}(v)$. Output
  $\sigma=(u,x,b)$ as the signature.
\end{enumerate}
\item[$\sigver(vk,\mu,\sigma)$] On input message $\mu \in \bit^{\alpha}$,
  verification key $vk$, signature $\sigma=(u,x,b)$:
\begin{enumerate}
\item Compute $\vecy \gets \ch.\chhash(x,\mu)$. 
\item Compute  $a^* \gets \evalfunc(g,\veca,(\vecy,b))$
\item 
Verify that the $\phtdf.\tdfprop(x)$\jnote{TODO: how to verify shortness}

$\phtdf.\tdfprop(u) \leq s$, and that $\phtdf.f_{pk,a^*,x}(u)=v$.
\end{enumerate}
\end{description}

\paragraph{Correctness.} Since $b=\weakprf(s,\vecy)$, we have that
$g(s,(\vecy,b))=1-b-b=1-2b=\pm 1$ whenever $b \in \bit$. Since this will
always be an invertible element of the ring $\calT=R_q$, we have that
$\phtdf.\tdfinv$ will invert successfully, and so verification will
succeed with all but negligible probability.


\subsection{Security}
\label{sec:security}

\paragraph{Security.} We now prove that the signature scheme in Section~\ref{sec:construction} is
  secure.
\begin{theorem} Let $\Adv$ be a PPT adversary which breaks eu-acma
  security of the scheme in Section~\ref{sec:construction} in time $t$
  with advantage $\epsilon$. Then there exists an $\Adv'$ which runs
  in time $t+\poly(\lambda)$ and breaks the security of one
  of the weak PRF, the chameleon hash function and the PHTDF with
  advantage at least $\epsilon/3$.
\end{theorem} 
\begin{proof}
We classify a successfully forging adversary into one of three
mutually exclusive categories, and show that each category of adversary can be
used for a successful attack on one of the cryptographic
primitives underlying our scheme.

Our reduction then proceeds by guessing uniformly at
random which of these three categories our adversary belongs to and
running the attack below corresponding to that category of adversary. It
is easy to see that we will succeed in breaking one of the underlying
cryptographic primitives with advantage at least $\epsilon/3-\negl(\lambda)$. 

To complete our proof, we proceed to describe these categories of adversaries and how we
use them to mount an attack on the underlying cryptographic primitives
that succeeds except with probability negligible in the underlying
security parameter $\lambda$.

 \paragraph{Weak PRF.} First, we consider the case that, conditioned on
 having forged successfully, the adversary outputs $\mu$,
  $\sigma=(u,x,b)$ such that $\weakprf(\vecz,(\chhash(x,\mu))=b$. For this
  case, we use $\Adv$ to help us succeed in the underlying $\weakprf$
  security game against some challenger. 

We change our behavior in the
  signature scheme's security game as follows:
\begin{enumerate}
\item
Instead of encoding the challenger's actual $\weakprf$ secret key (which we do not
know), we set the secret key $\vecz=(\vecs,\vecz_{k-1},\vecz_{k})$ in our scheme
  to a dummy secret key of $(\veczero,0,0)$ in $\siggen$, which will
  always allow us to sign regardless of the value of the
  $b=\weakprf(\vecs,\vecy)$. 

Since the $a_i$ output by $\phtdf.\gentrap$ are statistically close to
uniform, independent of the $\vecz$ encoded in them, and everything
else in the public key is completely independent of $\vecz$, this change is
  statistically indistinguishable. 
\item Given a query for message $\mu$, we send a query to the
  $\weakprf$ challenger, receiving back a uniform random
  $\vecy \in \calT^{w-1}$, $b=\weakprf(\vecs,\vecy)$ for some unknown
  $\vecs$. We then compute $x \gets \chinv(\mu,\vecy)$ as usual.
  Since $\vecy$ is uniform, this is distributed within negligible
  statistical distance of the manner it is chosen actual signature
  scheme.  Moreover, since by our condition on $\weakprf$ we will
  have that $g(\vecz,(\chhash(x,\mu),b))=1-z_{\kappa-b}-0=1$, we can invert
  successfully and sign as usual.
\end{enumerate}
As a result, our behavior in this game is statistically
indistinguishable from that in the real game.   At the end, if the adversary fails
  to output a valid signature, we choose a message $\mu^*$ and bit
  $b^*$ uniformly at random and send them to the $\weakprf$ challenger as our
  guess. Otherwise, if the adversary is successful and outputs a valid
  signature $\sigma=(u^*,x^*,b^*)$, we send that $\mu^*, b^*$ to the
  challenger.  Since the adversary will
  correctly predict $b$, we
  succeed in guessing the output of the $\weakprf$ at $\mu^*$.

Otherwise, a successfully forging $\Adv$ outputs $\mu^*$, $\sigma^*=(u^*,x^*,b^*)$ such
that $\weakprf(\vecz,(\chhash(x^*,\mu^*)) \neq b^*$. In this case, we
consider whether or not the forgery was achieved based on a hash
collision. 
\paragraph{Chameleon Hash Collision.} We consider the case that  $\Adv$ finds a hash
collision by successfully outputting $\mu^*,\sigma=(u^*,x^*,b^*)$ such that
$\chhash(x^*,\mu^*)=\chhash(x',\mu')$ for some $x',\mu'$ it already
queried on,  conditioned on a successful forgery and an unsuccessful weak
PRF prediction. We now show how to use $\Adv$ to
break the collision-resistance of the chameleon hash function
family. We change our behavior in the security game as follows:
\begin{enumerate}
\item In $\siggen$, we receive the evaluation key $ek$ for the
  chameleon hash function from a challenger instead of generating it
  (along with a trapdoor) ourselves. The public key itself remains
  exactly the same (although we no longer have $td$ in our secret
  key), so the adversary's view does not change.
\item In Step 1 of $\sigsign$, we instead sample $x \gets
\calX$, and then compute $\vecy = \chhash(x,\mu)$. By the uniformity
property of the collision-resistance hash function, this is within
statistically negligible distance of the manner $x$ and $\vecy$ are
evaluated in the actual scheme.
\end{enumerate}
Since our behavior is statistically indistinguishable from that in the
real game, $\Adv$ will still output a forgery such that 
$\chhash(x^*,\mu^*)=\chhash(x',\mu')$ for some previously output $x',
\mu'$. We can then send
$x',x^*,\mu',\mu^*$, succeeding in breaking the collision-resistance
of the chameleon hash function. 

\paragraph{PHTDF Collision When Punctured.} Finally, if $\Adv$ has
forged successfully, but has neither successfully predicting the weak
PRF output nor output  a successfull chameleon hash collision,
we can break collision-resistance when punctured (CRP)-security of the
PHTDF. We change our behavior in the security game as follows:
\begin{enumerate}
\item During $\siggen$, we choose a message $\mu'$ uniformly at
  random. Then, instead of choosing $v \gets \calV$, 
we invert the process,  sampling
  $u' \gets D_{\calU}$, $x' \gets \calX$. We then let
  $\vecy'=\ch.\chhash(x',\mu')$, $b=\weakprf(\vecs,\vecy')$, and compute
  $a' \gets \evalfunc(g,\veca,(\vecy',1-b))$. 

Finally, we set
  $v \gets f_{pk,a',x'}(u')$, and hold onto $u'$ to use in forming our
  collision. By the statisticial distributional
  equivalence of inversion property of the PHTDF, this is
  statistically indistinguishable from having sampled $v \gets \calV$
  as in the real security game. Moreover, we have that
  $g((\vecs,z^{(0)},z^{(1)}),(\vecy',(1-b)))=0$.
\end{enumerate}
If the adversary outputs a successful forgery
$(\mu^*,\sigma^*=(x^*,u^*,b^*))$, where
$\weakprf(\vecs,(\chhash(x^*,\mu^*))=1-b$, then because $\Adv$ has no
knowledge of the $x'$ we sampled above, with all but 
probability $1/\abs{\calX}=\negl(\lambda)$, $x^* \neq x'$. If they are in fact not equal, we can
output $u',x',\vecy',u^*,x^*,\vecy^*=\chhash(\mu^*,x^*)$ as our collision,
breaking the CRP security of the underlying PHTDF except with
probability negligible in $\lambda$. 

\end{proof}
\subsection{Efficient Evaluation of $g$}
\label{sec:weak-prf-evaluation}
\jnote{TODO: they say it's not self-contained, fix it and say
  more/needs more explanation}


\jnote{Reprove cited theorem from DM paper, also make sure }

Here we discuss how to homomorphically evaluate the
function
\[g(\vecz,(\vecy,b))=(1-z_{2-b})-\weakprf(\vecz,\vecy)\] 
using a cyclotomic ring-based instantiation of PHTDFs in a manner that
is (somewhat) efficient in terms of both time and noise growth, for
the specific case that $\weakprf$ is the $\lwr$ function. This
essentially boils down to efficiently evaluating the $\lwr$ function,
as once $b=lwr_{k,Q,2}$ has been homomorphically computed, the
remaining operations consist solely of additions. Our method of
evaluation is very close to that of Ducas and
Micciancio~\cite{DBLP:conf/eurocrypt/DucasM15}, but avoids any use of
the general lattice setting for efficiency.









% \begin{lemma}{Implicit in Proposition 6.2,~\cite{MilneANT}}
% Let $p$ be a (rational) prime, $\zeta$ a primitive $p^r$th root of
% unity for some $r\geq 1$. Then 
% \[\prod_{i \in \Z_{p^r}^{*}}(1-\zeta^i)=\Phi_{p^r}(1)=p\]
% \end{lemma}




\subsubsection{Lagrange Interpolation over
  $R_q$}\label{sec:lagr-interp-over}

To evaluate the rounding part of the $\lwr_{n,Q,2}$ function homomorphically,
we will need to be able to evaluate a function over $R_q$ (where $R=\O_{m}$) that sends $\zeta^i$ to
$0$ when $i \in [-Q/4,Q/4)$, and sends $\zeta^i$ to $1$ otherwise.    

% If $R_q$ were a finite field, we could immediately do so with Lagrange
% interpolation. However, it is known that the $2^k$th cyclotomic polynomials are
% reducible modulo every prime. 
% While we could simply work over an embedded subfield $P^{\ell}$ of
% sufficiently high degree $\ell$, this would either limit our choice of
% rings or require a significant loss in parameters. Instead, we show
% that Lagrangian interpolation will work over $R_{q}$ as long as $q$
% and $m$ are coprime. 

% \begin{lemma}\label{lem:prod_stuff}
% \[\prod_{i=1}^{m-1}(1-\zeta^i) = m\]
% \end{lemma}
% \begin{proof}
% We have that $X^{m}-1=(X-1)(X-\zeta)\ldots (X-\zeta^{m-1})$, so 
% $(X^{m}-1)/(X-1)= \sum_{i=0}^{m-1}X^{i}$. Evaluating this at $X=1$
% gives the required result.
% \end{proof}

% By multiplying each of the $m-1$ terms in the above product by
% $\zeta^{j}$, we get 
% \begin{corollary}\label{cor:invertibledenom}
% For any integer $j$,
% w\[\prod_{i \in [m], i \neq j}(\zeta^j-\zeta^i) = \zeta^{-j}m\]
% \end{corollary}

% As a result, as long as $m$ is invertible modulo $q$, the product of
% the differences will be invertible, so we can use Lagrange interpolation to compute the degree
% $m-1$ polynomial required to evaluate the
% rounding part of the $\lwr$ function. 

% \subsubsection{LWR Function Evaluation}
% \label{sec:lwr-function}

To homomorphically evaluate $\lwr_{n,Q,2}$ (where $Q=2^{\ell}=O(\lambda)$ for
$\ell=\ceil{\log_2(\lambda)}$), it will be easiest to work directly over the $2^{\ell}$th cyclotomic
ring $R=\O[2^{\ell}]$; this restricts us to odd moduli $q$ in the
actual PHTDF instantiation over $R_q$. The actual evaluation process will then
be as described in the work of Ducas and
Micciancio~\cite{DBLP:conf/eurocrypt/DucasM15}, using $q=3^\alpha$ for
some $\alpha$. We restate the
relevant result here for $\lwr$, recall again that evaluating $\lwr$
homomorphically 
over a PHTDF instantiated in the ring setting.
is morally equivalent to evaluating the decryption function. Note that
the conditions on parameter $s$ assumes the use of the  sampling ideas
found Section~\ref{sec:redtrapgrowth} to minimize the size of
parameter $s$, for modulus $q=3^{\alpha}$. 
\begin{theorem}[~\cite{DBLP:conf/eurocrypt/DucasM15}] Let $\phtdf$ be
  instantiated over $R_q$, where $R=\calO[m]$, where $q$ and $m$ are
  coprime. Then $\lwr_{n,Q,2}$ is
  admissible with parameter $s=O(n^{1.5}\log{Q})$, and can be
  evaluated with $O(n\log{Q})$ homomorphic operations, with the secret
  key $\vecz \in \Z_Q^{n}$ encoded in $\kappa-2=n\log{Q}$ functions
  $a_1, \ldots, a_{\kappa-2}$ with associated tags $\tilde{z}_1,
  \ldots, \tilde{z}_{\kappa-2}$. 
\end{theorem}

We finally have the following corollary for a concrete instantiation
using the PHTDF of Alperin-Sheriff (see
Section~\ref{sec:punct-homom-trapd}). 

\begin{corollary}[~\cite{DBLP:conf/eurocrypt/DucasM15}] Let $\phtdf$ be
  instantiated over $R_q$, where $R=\calO[m]$, where $q$ and $m$ are
  coprime. Then the above signature scheme is secure as long as both
  of the following hold:
\begin{itemize}
\item  $\lwr_{n,Q,2}$ is secure for $n,Q=O(\lambda)$
\item $\rsis$ is secure over $m$ with parameter
  $\beta=\tilde{O}(n^{7/2})$.
\end{itemize}
\end{corollary}

% For a weak PRF secret $\vecz \in \calT^{\kappa}$, for $i \in [kappa-2]$ we encode the $i$th coefficient
% $z_{i}$ as $\zeta^{z_{i}}$ in the $i$th trapdoor function. For the
% last two coefficients, recall
% that $z_{\kappa-1}=0$ and $z_{\kappa}=1$ in the actual scheme, and that these are only
% used in the very last step of computing the function $g$.


% Let $c_0,\ldots,c_{q-1}$ be the coefficients of the polynomial (over
% $R/qR$) that sends $\zeta^{q^2/4}, \ldots, \zeta^{3*q^2/4-1}$ to $1$,
% $\zeta^{qi}$ to 0 for $i \notin [q/4,3*q/4-1]$. Such a polynomial
% exists, and can be found efficiently via Lagrange interpolation (see
% Section~\ref{sec:lagr-interp-over}).

% We now show how to evaluate the LWR function on an input $\vecy \in
% \calT^{w-1}$ (where $w=n+1$) over a homomorphically
% encoded secret $\vecz$. In particular, for each $i$ we homomorphically compute 

% \[x=\prod_{j \in [n]}(\zeta^{s_j})^{y_j},\]
% viewing the exponentiation as the product of $y_j$ instances of
% $\zeta^{s_j}$, and evaluating  the entire product in the standard asymmetric manner~\cite{DBLP:conf/innovations/BrakerskiV14,DBLP:conf/crypto/Alperin-SheriffP14} to keep the
% noise growth quasi-additive. 

% $b=\sum_{i \in [0,\ldots,q-1]}c_ix^i



%%% Local Variables:
%%% mode: latex
%%% TeX-master: "newlattsigs"
%%% End:
