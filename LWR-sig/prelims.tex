\section{Preliminaries}
We write $[d]$  to denote the set of positive integers $\{1,
\ldots, d\}$. For an integer $q\geq 2$, we use $\Z_q$ to denote the
ring of integers modulo $q$, and somewhat abuse notation by also using
$\Zq$ to explicitly represent the
integers in $(-q/2,q/2]$. We define $\abs{x} \in \Z_q$ by taking the
absolute value of the representative in this range. 

We use $\otimes$ to denote the Kronecker product of two matrices. 

\subsection{Signatures}
\label{sec:signatures}
We briefly recall the standard definitions of digital signature schemes.
A \emph{signature scheme} $\sig$ is a triple
$(\siggen,\sigsign,\sigver)$ of PPT (probabilistic polynomial time)
algorithms, together with a message space $\calM=\calM_{\lambda}$. It
is correct if, for all messages $\mu \in \calM_{\lambda}$,
$\sigver(vk,\mu,\sigma)=1$ holds true, except with negligible
probability in $\lambda$ over the choice of $(sk,vk) \gets
\siggen(1^\lambda)$ and $\sigma \gets \sigsign(sk,\mu)$. 

We now recall the standard security definitions for digital signature
schemes. Existential unforgeability under adaptive chosen-message
attack, or eu-acma, is as follows: the adversary The challenger
generates keys $(vk,sk) \gets \siggen$ and sends $vk$ to $\Adv$. At
this point, $\Adv$ can adaptively requests signatures on messages
$\mu$, and the challenger responds with $\sigma=\sigsign(sk,\mu)$, for
each message.Finally,
  $\Adv$ outputs an attempted forged signature $(\mu^*, \sigma^*)$. In
  order to satisfy eu-acma security, the probability that $\mu^* \neq
  \mu_i$ for any $i \in [Q]$ and that
  $\sigver(vk,\mu^*,\sigma^*)=1$ accepts should be negligible in the
  security parameter $\lambda$. 

\subsubsection{Chameleon Hashing}
\label{sec:chameleon-hashing}


Chameleon hash functions, invented by Krawczyk and Rabin~\cite{DBLP:conf/ndss/KrawczykR00}, have been applied to signature
schemes for a number of purposes, notably for generic transformations
from statically-secure signatures (where the challenger receives the
messages to be signed before giving the adversary the verification
key) to adaptively-secure signatures (where the security game proceeds
as above). As our usage of chameleon hash functions is slightly
non-standard, we briefly recall their definitions, following the
variant definition of Ducas
and Micciancio specialized to their lattice-based
construction~\cite{DBLP:conf/crypto/DucasM14}. 

\begin{definition}A chameleon hash function family is a set of three algorithms
  $\ch=(\chgen,\chhash,\chinv)$ along with an efficiently computable
  input distribution $\calX_n$, $\calY_n$ for each integer $n$, where
  $\bot \notin \calY_n$. Except with negligible probability over the
  choice of $(ek,td) \gets \chgen(1^n)$, the following should hold for
  security.
\begin{description}
\item[Uniformity:] For a fixed message $\mu$, evaluation key $ek$,
  trapdoor $td$, 
  $(x \gets \calX_{n}, y \gets \chhash_{ek}(\mu,x))$ should be
  distributed within negligible statistical distance of $(x \gets
  \chinv_{td}(\mu,y), y \gets \calY_{n})$
\item[Collision Resistance:] Given only access to $ek$ and public
  parameters, it should be hard for any PPT algorithm $\calA$ to
  output $(\mu,r) \neq (\mu',r')$ such that
  $\chhash_{ek}(\mu,r)=\chhash_{ek}(\mu',r') \neq \bot$. 
\end{description}
\end{definition}

We also note that the Ducas and Micciancio paper provides an explicit
ring-based construction that is significantly more efficient in both
space and time than our main construction, and that this construction
has a very straightforward adaptation to general lattices and module lattices~\cite{DBLP:journals/dcc/LangloisS15}.


\subsection{Lattices}
\label{subsec:gaussians_lattices}

The main result of the paper in Section~\ref{sec:func-eval}
abstracts out the concrete lattice details in the form of Puncturable
Homomorphic Trapdoor Functions (PHTDFs); see
Section~\ref{sec:punct-homom-trapd} below. However, for our
complementary result on reducing the trapdoor growth for signature schemes
(Section~\ref{sec:redtrapgrowth}), we do need to recall some very basic results.

Following Gentry et al.~\cite{DBLP:conf/stoc/GentryPV08}, for integers $n
\geq 1$, modulus $q \geq 2$, we define the $m$-dimensional lattice specified by an ``arity check'' matrix $\matA \in \Zq^{n
  \times m}$:
\[\Lambda^{\perp}(\matA) = \set{ \vecx \in \Z^m: \matA\vecx = \veczero
  \in \Zq^{n} } \subseteq \Z^m,\]and for $\vecy$ in the subgroup of
$\Zq^{n}$ generated by the columns of $\matA$, we define the
coset
\[\Lambda_{\vecy}^{\perp}(\matA) = \set{ \vecx \in \Z^m: \matA\vecx =
  \vecy \bmod q} = \Lambda^{\perp}(\matA)+\bar{\vecx},\]
  where
 $\bar{\vecx} \in \Z^m$ is an arbitrary solution (not necessarily
 short) to
 $\matA\bar{\vecx}=\vecy$.

\paragraph{``Gadget'' matrix $\matG$}
We recall the gadget matrix $\matG$ defined by Micciancio and
Peikert~\cite{DBLP:conf/eurocrypt/MicciancioP12}. We focus on the case
that the modulus $q=2^{k}$ for ease of analysis.

We define
\[\vecg^{t}=[1,2,2^2,\ldots,2^{k-1}] \in \Z_q^{1 \times k}\]
Then we have that 
\[\matG = \matI_{n}
\otimes \vecg^{t} \in \Zq^{n \times nk}\] 

While $\matG$ is used by Miccancio and Peikert to sample discrete
Gaussians, in this work we only use it for computing a simpler
distribution (see Section~\ref{sec:redtrapgrowth} for details).



\paragraph{The SIS problem.} For $\beta > 0$, the \emph{short integer
  solution} problem $\sis_{n,q,\beta}$ is an average-case version of the
approximate shortest vector problem on $\lamperp(\matA)$. Given a
uniformly random matrix $\matA \in \Z_q^{n \times m}$ for any
$m=\poly(n)$, the problem is to find a nonzero vector $\vecz \in \Z^m$
such that $\matA\vecz=\veczero \bmod{q}$ and $\length{\vecz} \leq
\beta$. For $q \geq \beta\sqrt{n}\omega(\sqrt{\log{n}})$, it has been
shown that solving this problem with non-negligible success
probability over the random choice of $\matA$ is at least as hard as
probabilistically approximating the classic Shortest Independent Vectors
Problem (SIVP) on $n$-dimensional lattices to within
$\tilde{O}(\beta\sqrt{n})$ factors in the \emph{worst
  case}.~\cite{ajtai04:_gener_hard_instan_lattic_probl,DBLP:journals/siamcomp/MicciancioR07,DBLP:conf/stoc/GentryPV08}. Analogous
worst-case reductions exist for the more general case of module lattices, where
$\matA \in R_q^{d \times m}$ for some arbitrary ring of integers $R$
algebraic number field $K$~\cite{DBLP:journals/dcc/LangloisS15}, which
essentially includes the above results as a special case.




\subsection{Subgaussian Random Variables}
\label{sec:subgauss}

To analyze our distribution in Section~\ref{sec:redtrapgrowth}, we
make use of the notion of \emph{subgaussian} random variables.  (For
further details and full proofs,
see~\cite{vershynin12:_compr_sensin_theor_applic}.)  A random
vector~$\vecx$ is subgaussian with parameter~$r > 0$ if for all $t \in \R$
and all (fixed) real unit vectors $\vecu$, its (scaled)
moment-generating function satisfies
$\E\bracks{\exp(\inner{\vecu,\vecx})} \leq \exp(C r^{2} t^{2})$ for an
absolute constant $C$ (for our application, we may take $C=1$).  By a Markov
argument, for all $t \geq 0$, we have
\begin{equation}
  \label{eq:subgaussian-tails}
  \Pr\bracks{\length{\vecx} \geq t} \leq 2 \exp(- t^{2}/r^{2}).
\end{equation}
 Any
$B$-bounded centered random vector~$\vecx$ (i.e., $\E[\vecx]=\veczero$ and $\abs{X}
\leq B$ always) is subgaussian with parameter $B$.

We recall the following additional properties of subgaussian vectors
$\vecx$~\cite{vershynin12:_compr_sensin_theor_applic}.
\begin{description}
\item[Homogeneity:] If $\vecx$ is subgaussian with
parameter~$r$, then $c\vecx$ is subgaussian with parameter $c \cdot r$ for
any constant $c \geq 0$.
\item[Pythagorean
  additivity:] if~$\vecx_{1}$ is subgaussian with parameter~$r_{1}$,
and~$\vecx_{2}$ is subgaussian with parameter~$r_{2}$ conditioned on
\emph{any} value of~$\vecx_{1}$ (in particular, if $\vecx_2$ is
subgaussian with parameter~$r_2$ and independent of $\vecx_{1}$), then $\vecx_{1} + \vecx_{2}$ is subgaussian with parameter
$\sqrt{r_{1}^{2} + r_{2}^{2}}$.
\item[Euclidean Norm]: 
  Let $\vecx \in \R^{n}$ be a random vector with independent
  coordinates that are subgaussian with parameter~$r$.  Then for some
  (small)   universal constant $0 < C$, we have $\Pr[\length{\vecx}_{2} > C
  \cdot r\sqrt{n}] \leq 2^{-\Omega(n)}$
\end{description}

\subsection{Weak Pseudorandom Functions and Learning with Rounding}
\label{sec:weak-pseud-funct}
Here we give a basic definition of weak pseudorandom
functions~\cite{DBLP:conf/crypto/DamgardN02}.  

More formally, a weak pseudorandom
function family (outputting a single bit)
$\weakprf: \bit^{\lambda} \times \bit^{m} \to \bit$ is considered
secure if no probabilistic polynomial-time adversary can distinguish a
member of the family $f_{\veck}:\bit^{m} \to \bit, f_{\veck}:=\weakprf(\veck,\cdot)$ (where $\veck
\gets \bit^{\lambda}$ uniformly at random) 
from a truly random function with advantage greater than
$\negl(\lambda)$, given that it can observe
\[(x_1, f_{\veck}(x_1)), \ldots, (x_m, f_{\veck}(x_m))\] for any $m \in \poly(\lambda)$,
where each $x_1, \ldots, x_m$ is sampled uniformly at random from
$\bit^{m}$. 

A concrete candidate weak pseudorandom function family is the learning with
rounding function family
($\lwr_{n,Q,p}$)~\cite{DBLP:conf/eurocrypt/BanerjeePR12}.
Functions in the family are indexed by a secret key $\vecs \in
\Z_Q^n$. For a given secret key $\vecs$, the function  is defined as 
\[\lwr_{n,Q,p}(\veca)=\round{\tfrac{p}{Q}\inner{\veca,\vecs}},\]
where $\round{\cdot}$ denotes rounding to the nearest integer. 

$\lwr$ has been shown to be a weak pseudorandom function under the
better-known $\lwe$ assumption with discrete Gaussian noise terms (and hence on worst-case shortest
vector problems on lattices) in a number of different results~\cite{DBLP:conf/eurocrypt/BanerjeePR12,DBLP:conf/crypto/AlwenKPW13,cryptoeprint:2016:589,DBLP:conf/tcc/BogdanovGMRR16,DBLP:conf/asiacrypt/BaiLLSS15}. While
all of these results require the ratio $Q/p$ to grow with the number
of samples revealed (meaning that hardness for non-a-priori bounded
$m=\poly(\lambda)$ requires assuming the shortest vector problem is hard
to approximate to within a superpolynomial ratio), there is a
reduction~\cite{DBLP:conf/tcc/BogdanovGMRR16} with a sample loss ratio of $Q/p$ in
security from $\lwe$ with bounded uniform error to $\lwr$. This latter result
strongly suggests that $\lwr$ remains a weak pseudorandom function
for any $Q/p=\Omega(\sqrt{n})$, and that the weaker reductions from
$\lwe$ with Gaussian error are likely all artifacts of the proof
techniques used.

\subsection{Puncturable Homomorphic Trapdoor Functions}
\label{sec:punct-homom-trapd}
We recall Alperin-Sheriff's definition of Puncturable Homomorphic
Trapdoor Functions (PHTDFs). 

\begin{description}
\item[$pk \gets \tdfgen(1^{\lambda})$] takes
  as input a security parameter $\lambda$, which for a concrete
  instantiation implictly defines parameters for a ring $\calT$ representing a tag space, a trapdoor space
  $\calR$, a tagged function space $\calA$, an index space $\calX$, an
  input space $\calU$ and an output space $\calV$, and then generates
  the public key for the PHTDF.  $\calR$ and $\calU$ are associated with
  parameterized efficiently sampleable distributions
  $D_{\calR,\beta}, D_{\calU,s}$, with the distribution details
  depending on the instantiation. 
\item[$(a,r) \gets \gentrap(pk,t)$] generates a \emph{trapdoor} $r \gets
  D_{\calR}$ for the $(pk,a)$, with $t$ the tag \emph{associated
  with} $a,r$. We need the
  distribution of $a$ to be statistically close to uniform over
  $\calA$. 
\item[$t \gets \tdftag(pk,a,r)$] is an auxiliary function which
  outputs the tag $t$ associated with $a$ and $r$ is a trapdoor
  for $(pk,a)$. 
\item[$f_{pk,a,x} : \calU \to \calV$] is a deterministic 
function indexed by $pk,x \in \calX,a \in \calA$. 
\item[$\tdfinv_{r,pk,a,x,s} : \calV \to \calU$] is a trapdoor-inverter
  indexed by $x \in \calX$, $r \in \calR$ and $pk,a \in \calA$. If
  $f_{pk,a,x}$ is not injective, then $\tdfinv$ is a probabilistic
  function, and the parameter $s \in \R$ relates to the noise level $\tdfprop(u)$ of
  the inverse $u$ output by $\tdfinv$; in particular, we want
  $\tdfprop(u) \leq s$. We require that $\tdfprop(r)=\beta$ should be small
  enough to allow $\tdfinv$ to invert with parameter $s$ when the tag
  $t$ associated with $a,r$ is \emph{invertible} over the ring
  $\calT$. If $t$ is not invertible, then the trapdoor is considered
  \emph{punctured}, and $\tdfinv$ outputs $\bot$.
\item[$r^* \gets
  \evaltd(g,\{(a_i,r_i)\}_{i \in [\kappa]},\vecy),
  a^* \gets \evalfunc(g,\{a_i\}_{i \in [\kappa]},\vecy)$] are
  deterministic trapdoor/function homomorphic evaluation algorithms, respectively.
  The algorithms take as input some function $g: \calT^{\kappa} \times
  \calT^{w} \to
  \calT$, a vector $\vecy \in \calT^{w}$, as well as functions $a_i
  \in \calA$ with associated trapdoors $r_i \in \calR$. The outputs are $r^* \in \calR$ and
  $a^* \in \calA$.

Let $\vect \in \calT^{\kappa}$ be a vector
such that $t_i$ is the tag associated with $a_i,r_i$. We refer to $g$ as
  \emph{admissible} with parameter $s$ on $\vect$ if for any $v
  \in \calV,\vecy \in \calT^{w}$ such that $g(\vect,\vecy)$ is invertible,
$\tdfinv_{r^*,pk,a^*,x,s}(v)$ successfully outputs $u$ such that with
overwhelming probability, $\tdfprop(u) \leq s$.
\end{description} 

\paragraph{Security Properties.} 

The following should hold for $pk \gets \tdfgen(1^\lambda)$,
trapdoor and function pair $(r,a)$ with an invertible tag $t$:
\[(pk,r,a,x,u,v) \statind (pk,r,a,x,u',v')\] where $x \in \calX$ is arbitrary, $u \gets
D_{\calU,s}$, $v := f_{pk,a,x}(u)$, $v' \gets \calV$ and $u' \gets
\tdfinv_{r,pk,a,x,s}(v')$.


The security game between an adversary $\Adv$ and a challenger $\calC$
is parameterized by a security parameter $\lambda$, as well as a
function $g: \calT^{\kappa} \times \calT^{w} \to \calT$ such that $g$
is admissible with some parameter $s$ on some subset of tags
$\calS \subseteq \calT^{\kappa}$
\begin{enumerate}[itemsep=1pt]

\item $\calC$ runs $pk \gets \tdfgen(1^{\lambda})$ and then
  computes $(a_i, r_i) \gets \gentrap(pk,\vect)$ for each $i \in
  [\kappa]$.  $\Adv$ is given $pk$ and $\{a_i\}$.
\item $\Adv$ may make (a polynomial number of) inversion queries, sending some $v \in
  \calV$, $x \in \calX$ and some $\vecy \in \calT^{w}$ such that
  $g(\vecs,\vecy)$ is \emph{invertible}. $\calC$ computes
  $r' \gets \evaltd(g,\{(a_i,r_i)\},\vecy)$ as well as $a' \gets
  \evalfunc(g,\{a_i\},\vecy)$, samples $u \gets \tdfinv_{r',pk,a',x,s}(v)$ and returns
$u$ to $\Adv$. 
\item $\Adv(1^\lambda)$ outputs tag sets $\vecy^{(1)}, \vecy^{(2)} \in \calT^{w}$ which satisfy\\
  $g(\vect,\vecy)=g(\vect,\vecy')=0$, as well as
  $u^{(1)},u^{(2)}, x^{(1)}) \neq x^{(2)}$, and wins if
  \[f_{pk,a^{(1)},x^{(1)}}(u^{(1)})=f_{pk,a^{(2)},x^{(2)}}(u^{(2)}),\] where
  $\tdfprop(u^{(1)}),\tdfprop(u^{(2)}), \tdfprop(x^{(1)}),\tdfprop(x^{(2)})  \leq s$ and $a^{(b)} \gets
  \evalfunc(g,\{(a_i)\}, \vecy^{(b)})$ for $b \in \{1,2\}$.
\end{enumerate}

We say the PHTDF satisfies
$(\epsilon=\epsilon(\lambda),t=t(\lambda),g,\calS)$-collision
resistance when punctured (CRP) security if every PPT adversary taking
at most time $t$ has success probability at most $\epsilon$ in this
game.

\subsubsection{Concrete Instantiation}
\label{ref:concrete_inst}
The instantiation in~\cite{DBLP:conf/pkc/Alperin-Sheriff15} is written
in terms of general lattices, but as mentioned in that work, can
easily be instantiated over rings or
modules~\cite{DBLP:journals/dcc/LangloisS15}. We briefly recall the
relevant results on security, leaving tag instantiation descriptions to later in
the paper.

\begin{theorem}[~\cite{DBLP:conf/pkc/Alperin-Sheriff15}]
Let $g$ be admissible with parameter $s$. 
If there exists an adversary $\Adv$ breaking
$CRP_{\epsilon,t,g,\calS}$ security of the PHTDF, then there exists
$\Adv'$ running in time $t$ that solves $\sis_{n,q,\beta}$ with
advantage $\epsilon-\negl(\lambda)$ for $\beta=O(s^2\sqrt{n
  \log{q}})$. 
\end{theorem}

We will also need to recall the growth rate of the $\tdfprop(r)$ for
the trapdoors used in this instantiation as a result of homomorphic
operations. In particular, we have 
\begin{description}
\item[Homomorphic addition] of $a_1$
and $a_2$ with trapdoors $r_1, r_2$ induces a new trapdoor $r^*$ with 
$\tdfprop(r^*)=\tdfprop(r_1)+\tdfprop(r_2)$ 
\item[Homomorphic Multiplication] of $a_1, a_2$ with trapdoor $r_1,
  r_2$, tags $t_1,t_2$ induces trapdoor $r^*$ with 
\[\tdfprop(r^*)=\tdfprop(r_1)\matG^{-1}(a_2)+\tdfprop(t_1)\tdfprop(r_2)\]
A key trick with homomorphic multiplication is to chain them together
in a left-associative manner, causing a quasi-additive growth in the trapdoors~\cite{DBLP:conf/innovations/BrakerskiV14,DBLP:conf/crypto/Alperin-SheriffP14}.
\end{description}
%%% Local Variables:
%%% mode: latex
%%% TeX-master: "newlattsigs"
%%% End: