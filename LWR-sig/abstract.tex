The Boyen-Li signature scheme [Asiacrypt'16] is a major theoretical breakthrough. Via a clever homomorphic evaluation of a pseudorandom function over their verification key, they achieve a reduction loss in security linear in the underlying security parameter and entirely independent of the number of message queries made, while still maintaining short signatures (consisting of a single short lattice vector). All previous schemes with such an independent reduction loss in security required a linear number of such lattice vectors, and even in the classical world, the only schemes achieving short signatures relied on non-standard assumptions. 

We improve on their result, providing a verification key smaller by a linear factor, a significantly tighter reduction with only a constant loss, and signing and verification algorithms that could plausibly run in about 1 second. Our main idea is to change the scheme in a manner that allows us to replace the pseudorandom function evaluation with an evaluation of a much more efficient weak pseudorandom function.

As a matter of independent interest, we give an improved method of randomized inversion of the $\matG$ gadget matrix [MP12], which reduces the noise growth rate in homomorphic evaluations performed in a large number of lattice-based cryptographic schemes,  without incurring the high cost of sampling discrete Gaussians. 