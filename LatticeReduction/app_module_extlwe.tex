\section{Extended-LWE Over Module Lattices}
\label{sec:extended-lwe-over}


Here we show that an analogous version of the $\extlwe$ reduction
found in~\cite{DBLP:conf/stoc/BrakerskiLPRS13} holds over module
lattices as well. In order to prove this theorem, we follow the
path of Brakerski et al.~\cite{DBLP:conf/stoc/BrakerskiLPRS13}.  We
first prove the reduction to the intermediate problem
``first-is-errorless-M-LWE'' and then proceed to prove the reduction
from ``first-is-errorless-M-LWE.'' to $\extmlwe$.

\subsection{Quality of Set}
\label{sec:set-quality}
Following Brakerski, et al, we define the \emph{quality} of a set
$\calZ \subseteq \Z^{m}$.

\begin{definition}\label{def:quality} For a real $\xi > 0$ and a set $\calZ \subseteq
  K^w$,
  we say that $\calZ$ is of \textnormal{quality} $(\xi,t)$ if given
  any $\vecz_1, \ldots, \vecz_t \in \calZ$, we can efficiently
  construct a unimodular matrix $\matU \in \Z^{w \times w}$ such that,
  if $\matU' \in \Z^{w \times (w-t)}$ is the matrix obtained from
  $\matU$ be removing its $t$ leftmost columns, then all the columns
  of $\matU'$ are orthogonal to $\text{span}(\vecz_1,\ldots,\vecz_t)$
  and its largest singular value is at most $\xi$.
\end{definition}


\subsection{First-Is-Errorless-LWE}
\label{sec:first-errorless-lwe}

As in the previous work, we have the intermediate step of
``first-is-errorless'' $\mlwe$, in which the first inner product
contains no error. The proof for this general case essentially follows
that of Brakerski et al., with a few important differences stemming
from the form of $R$ will not in general be a Euclidean domain or
even a principal ideal domain. This requires some careful proof
restructuring.
\begin{definition}\label{def:first-errorless-mlwe}
  Let $K$ be an algebraic number of field, $R=\O_{K}$ be its ring of
  integers. Let $q>1,d\geq 2$ be integers, and let $\psi$ be a
  distribution over $R$. Then the
  ``first-is-errorless'' version of $\mlwe$ requires the adversary to
  distinguish between two cases. In the first case, the adversary
  receives $\matA \in R_q^{d \times w},\vecb = \matA^t\vecs+\vece \in
  R_q^w$, where $\vecs \gets R_q^d$ uniformly at
  random and $\vece \gets (0, \vece')$, where $\vece' \gets
  \psi^{w-1}$.  In the second case, $\vecb \gets R_q^m$ uniformly at random.
\end{definition}

We first prove a brief lemma relating to the probability that the
greatest common divisor of $qR$ and the 
principal ideals generated  by $d$
elements chosen uniformly at random modulo $R_q$  is in fact all of
$R$ (making all of these ideals coprime). 
\begin{lemma}\label{lem:prob-relprime}
Let $q=\prod_{i \in
    [\ell]}p_i^{e_i}$. Let $R=\calO_{m}$ and write $m=\bar{m}\cdot
  p_i^{\kappa_i}$, where $p \nmid \bar{m}$. Let $f_i$ be the
  multiplicative order of $p_i$ modulo $\bar{m}$. Let $a_1, \ldots,
  a_d \gets R_q$ uniformly at random (strictly speaking, elements in
  $R$ from
  some canonical set of representatives of $R_q$). Then we have that 
$R=\gcd(qR,\inner{a_1},\ldots,\inner{a_d})=qR+\inner{a_1}+\ldots+\inner{a_d}$,
except with probability at most
\begin{equation}
\label{eq:upp-bound-lin-ind}\sum_{i \in
  \ell}\left(\frac{\varphi(\bar{m})}{f_i}p_i^{-d\cdot f_i} \right)
\end{equation}
\end{lemma}
\begin{proof}
  We have that the sum of all the ideals above will be equal to $R$
  unless $a_1, \ldots, a_d$ all lie in some prime ideal $\frakp_i$ of
  $qR$.  Recall from~\ref{par:ideal-fact} that a prime ideal
  $\frakp_{i,j}$ dividing $p_iR$ has norm
  $\length{R/\frakp_{i,j}}=p_i^{f_i}$, the probability of this event
  is at most $p_i^{-d\cdot f_i}$. Equation~\ref{eq:upp-bound-lin-ind}
  then follows from a simple union bound.
\end{proof}


We now prove the reduction.
\begin{lemma}\label{lem:first-errorless}
  Let $d \geq 2$, $w$, $q > 1$, $R=\calO_{m}$, $\psi$ be an error distribution
  over $R$. There is an efficient (transformation) reduction
  from $\mlwe_{R,d-1,w-1,q,\psi}$ to the first-is-errorless variant of
  $\mlwe_{R,d,w,q,\psi}$ that reduces the advantage by at most
$\sum_{i \in
  \ell}\left(\frac{\varphi(\bar{m})}{f_i}p_i^{-d\cdot f_i} \right)$, where
  $f_i$ and $\bar{m}$ are as defined in Lemma ~\ref{lem:prob-relprime}.
\end{lemma}

\begin{proof}
  The reduction first chooses $\veca'\gets R_q^d$. 
If we have that $qR+\inner{a'_1}+\ldots+\inner{a'_d} \subset R$, the reduction
aborts. Otherwise, it finds, via a generalization of the
  Euclidean algorithm~\cite{extendeuclidideals}, a $d \times d$
  invertible matrix  $\matR \in R_q^{d \times d}$ such that
  $\matR\veca'=(1,0,\ldots,0)^t$, and sets $\matU = \matR^{-1}$.  

It also picks a uniform element $s_0 \in
R_q$. It then outputs $(\veca', s_0)$  as the first sample. For
each of the $w-1$ remaining samples, it first samples $(\veca,b)$ from
the given $\mlwe$ oracle, picks a fresh uniformly random $d \in
R_q$, and outputs $(\matU(d|\veca), b+ (s_0\cdot d)$, where the
vertical bar denotes concatenation. 

Conditioned on not aborting, we will verify that the
output is distributed according $\mlwe_{R,d,m,q,\psi}$. Combining this
with the bound in Lemma~\ref{lem:prob-relprime} will give our desired result.

We now verify correctness, conditioned on not aborting. First, we consider the
case that
$(\veca,b)$ samples are uniform and independent. In this case it is
easy to see that the outputs remain uniform, since $(d|\veca)$ is
uniform, $\matU$ is invertible, and each $b$ is uniform and independent of
everything else. Secondly, if the samples are from $\mlwe$ with
respect to a secret $\vecs$, it is easy to verify that the outputs are
distributed according to the first-is-errorless variant
$\mlwe_{R,d,w,q,\psi}$ with secret $\vecs'=\matR^{t}(s_0|\vecs)
\bmod{q}$. Since $\matU$ is invertible modulo $q$ and $(s_0|\vecs)$ is
uniform, so is $\vecs'$, which proves correctness.
\end{proof}

By iterating this reduction and applying a union bound, we immediately have the following
corollary which gives a reduction to a first-$t$-are-errorless variant
of $\mlwe$. We use this corollary in the next section to reduce to
$\extmlwe$ with $t$ hints.
\begin{corollary}\label{lem:first-t-errorless}
  Let $d > t \geq 0$, $w$, $q > 1$,  $R=\calO_{m}$, $\psi$ be an error
  distribution over $R$. There is an efficient (transformation) reduction from
  $\mlwe_{R,d-t,w-t,q,\psi}$ to the first-$t$-are-errorless variant of
  $\mlwe_{R,d,w,q,\psi}$ that reduces the advantage by at most
  $t \sum_{i \in
  \ell}\left(\frac{\varphi(\bar{m})}{f_i}p_i^{-(d-t)\cdot f_i} \right)$, where
  $f_i$ and $\bar{m}$ are as defined in Lemma ~\ref{lem:prob-relprime}.
\end{corollary}

In order for our reduction to ``first-is-errorless'' $\mlwe$ to be
meaningful when $d-t$ is much smaller than linear in the underlying
security parameter $\kappa$ (as it will be in the case we are
interested in for our reduction to $\mlwr$), we will require that all
of the prime ideals of $qR$ have norms superpolynomially large in the
security parameter (which makes the reduction in advantage
negligible). This is the case for many if not most practical choices
of $R$ and $q$. Similar conditions were required by Ducas and
Micciancio~\cite{DBLP:conf/crypto/DucasM14} for the ring used in their
signature scheme, although they were significantly further restricted
because they needed the prime ideals to be exponentially large in the
security parameter. As in their work, for a concrete example of
choices of $R$ and $q$ for which this reduction is meaningful, we can
take $R$ to have index $2^{k}$ and let $q=3^{\ell}$. In fact, for $R$
of index $2^{k}$, noting that $\Z_{2^k}^* \equiv \Z_2 \oplus
\Z_{2^{k-2}}$, we have asymptotically by the prime number theorem for
arithmetic progressions~\cite{soprounov2010short} that the density
of all primes (and their powers) that are admissible as prime factors of
the modulus $q$ for which the reduction is meaningful is $1-1/2^{k-1}$

\subsection{First-Is-Errorless to Extended-LWE}
\label{sec:first-errorl-extend}

We now show a reduction from ``first-t-are-errorless'' $\mlwe$ to
$\extmlwe$. The reduction primarily follows that of Brakerski et al.,
with slight differences to account for the multiple samples, as well
as a correction of some small mistakes in the proof therein. 
\begin{lemma}
Let $\calZ \subseteq K^d$ be of quality $(\xi > 0,t)$. Then for any $n, q
> 1, \epsilon \in (0,1/2)$, $\alpha, r \geq
(\ln(2w(1+1/\epsilon))/\pi)^{1/2}/q$, $t < d$, there is a (transformation)
reduction from the first-$t$-are-errorless variant of
$\mlwe_{R,d,w,q,\alpha}$ to
\\$\extmlwe_{R,d,w,q,(\alpha^2\xi^2+r^2)^{1/2},\calZ,t}$ that reduces
the advantage by at most $16\epsilon$. 
\end{lemma}

\begin{proof}
The reduction proceeds as follows. Given $\vecz_1, \ldots, \vecz_t \in
\calZ$, we compute the unimodular $\matU \in \Z^{d \times d}$ that can
be efficiently computed as in Definition~\ref{def:quality}, and let
$\matU' \in \Z^{d \times (d-t)}$ be the matrix formed by removing the
  $t$ leftmost columns of $\matU$. We receive $m$ samples $(\matA \in
  \R_q^{d \times w}, \vecb \in \R_q^{w})$ from our
  ``first-$t$-errorless'' $\mlwe$ oracle. We
  then sample 
$\vecf \gets D_{\alpha(\xi^2\matI-\matU\matU^t)^{1/2}}$. The above distribution is well-defined since
  $\xi^2\matI-\matU'\matU'^t$ is a positive semidefinite matrix
  according to our assumption on $\matU$. 

The reduction then sets $\bar{\vecb}=\matU\vecb+\vecf$, and samples 
$\vecc$ from the discrete Gaussian distribution
$D_{q^{-1}R^{w}-\bar{\vecb},r}$. It then outputs 
\[(\matA'=\matA\matU^t,\vecb'=\bar{\vecb}+\vecc,(\trace(\inner{\vecz_i,\vecf+\vecc}))_{i
  \in [t]}) \in R_q^{d \times w} \times R_q^{w} \times \Q_q^{t}\]

We now analyze correctness of the reduction. First, we have that
$\matU$ is invertible, so that $\matA'=\matA\matU^{t}$ is uniform
solely over the choice of $\matA$. For the remainder of the proof we
condition on any fixed value of $\matA'$ and focus on analyzing the
distribution of the second and third components of the output.

First, consider the case that the samples received by our reduction
came from the first-$t$-are-errorless variant of $\mlwe$. Then we have
that 
\[\bar{\vecb}=\matU\vecb+\vecf=\matA'^t\vecs+\matU\vece+\vecf \]

We have that $\matU\vece$ is distributed as a continuous Gaussian
$D_{\alpha\matU}$, so by additivity of Gaussians, we have that
$\matU\vece+\vecf$ is distributed as a spherical continuous Gaussian
$D_{\alpha\xi}$. Since $\matA'^{t}\vecs \in R_q^w$, we have
that $\vecc$ is in fact distributed according to
$D_{q^{-1}R^{w}-\bar{\vecb},r}$.  As a result, by
Lemma~\ref{lem:latt-facts}, 
$\matU\vece+\vecf+\vecc$ is within statistical distance $8\epsilon$ of
$D_{q^{-1}R^{w},(\alpha^2\xi^2+r^2)^{1/2}}$, which shows that $\vecb'$
is distributed correctly. For the third component, since
the first $t$ elements of $\vece$ are zero and the last $d-t$
columns of $\matU$ are orthogonal to each $\vecz_i$, we have that 
\[\inner{\vecz_i,\vecf+\vecc}=\inner{\vecz_i,\matU\vece+\vecf+\vecc},\]
so the third component gives the inner product of the noise with each
vector $\vecz_i$, as desired. 

Finally, we consider the case where the samples received by our
reduction are truly uniform. Then since $\vecb$ is truly uniform and
independent, we can view it as being distributed as
$\vecb=\tilde{\vecb}+\vece$, where $\tilde{\vecb} \in R_q^{w}$
is truly uniform and $\vece$'s first $t$ coordinates are $0$, and the
remaining coordinates are chosen independently from $D_{\alpha}$. Then
we have that $\vecb'=\matU\tilde{\vecb}+\matU\vece+\vecf+\vecc$, where
$\matU\vece+\vecf+\vecc$ is distributed exactly as in the case
above. Since $\matU$ is unimodular, we have that
$\matU(\tilde{\vecb}+\vece)$ is uniform and independent over
$R_q^{w}$. Finally, since $\matU\vece+\vecf+\vecc$ is distributed
exactly as above, we also have that the third component output is
distributed correctly, as needed.
\end{proof}