\iflncs\section{Background and Definitions}
\label{sec:backgr-defin}
\fi


\subsection{Full Algebraic Number Theory Background}
\label{sec:algebr-numb-theory}



\paragraph{Cyclotomic Fields and Rings.}
For a positive integer $m$, the $m$th
\emph{cyclotomic number field} is $K=\Q(\zeta_{m})$, where $\zeta_m$ is some fixed arbitrary primitive $m$th root of unity (for each $m$,
we view it abstractly and not as any particular such root).  We denote
the ring of integers of $K$ as $\O_{K} = \Z[\zeta_{m}]$, and refer to
it as the $m$th cyclotomic ring. 

A cyclotomic number field
$K=\Q(\zeta)$ degree $n$ has exactly $n$ ring embeddings $\sigma_i: K
\to \C$, which can be defined by $\sigma_i(\zeta)=\zeta_i$ for $i \in
\Z_m^*$. The \emph{canonical embedding} is then defined as \[\sigma(x)
= (\sigma_1(x), \ldots, \sigma_{m-1}(x)) \in \C^{n}.\] We also define
the trace $\trace: K \to \Q$ and the (field) norm $N: K \to \Q$ as
$\trace(x)=\sum \sigma_j(x)$, $N(x)=\prod_{j}\sigma_j(x)$, and recall
that the trace is a ``universal'' $\Q$-linear function, in the sense
that  any linear function $L: K \to \Q$ can be expressed as
$L(a)=\trace(r\cdot a)$ for some fixed $r \in K$. 

% The minimal polynomial of $\zeta_{m}$
% over $\Q$ is the~$m$th \emph{cyclotomic polynomial}
% $\Phi_{m}(X)=\prod_{i \in \Z_{m}^{*}} (X-\omega_{m}^{i}) \in \Z[X]$,
% where $\omega_{m}=\exp(2\pi\sqrt{-1}/m) \in \C$ is the principal $m$th
% complex root of unity, and the roots $\omega_{m}^{i} \in \C$ range
% over all the \emph{primitive} complex $m$th roots of unity.
% Therefore, $\O_{m}$ is a ring extension of degree $n=\varphi(m)$ over
% $\Z$.  (In particular, $\O_{1}=\O_{2}=\Z$.)  Clearly, $\O_{m}$ is
% isomorphic to the polynomial ring $\Z[X]/\Phi_{m}(X)$ by
% identifying~$\zeta_{m}$ with~$X$, and has the ``power basis''
% $\set{1,\zeta_{m}, \ldots, \zeta_{m}^{n-1}}$ as a $\Z$-basis. Note
% that for $m\in \{1,2\}$, we have that $K=\Q$ and $R=\Z$.


\paragraph{Ideals.}\label{par:ideals} An (integral) \emph{ideal} $I$ of $R$ is an
additive subgroup of $R$ that is closed under multiplication by every
element of $R$. We denote the smallest ideal of $R$ containing the set
$S$ by $(S)$, and by $R/I$ the set of equivalence classes $x+I$ of $R$
modulo $I$. The norm of a (nonzero) ideal is the number of elements in
$R/I$.

A \emph{prime ideal} $I \subseteq R$ is such that if $ab \in I$,
then at least one of $a$ and $b$ is also in $I$. Over the ring of
integers of any algebraic number field, any ideal of $R$ can be
represented uniquely as a product of prime ideals.

\paragraph{Factorization of Ideals.}\label{par:ideal-fact} Let $q=p^r \in \Z$ be a prime power.  In the $m$th cyclotomic ring
$R=\O_{m} = \Z[\zeta_{m}]$ (which has degree $n=\varphi(m)$
over~$\Z$), the ideal $pR$ factors into prime ideals as follows. First
write $m = \bar{m}\cdot p^{k}$ where $p \nmid \bar{m}$. Let the
ramification index 
$e = \varphi(p^{k})$, and let the inertia degree~$d$ be the multiplicative order of $p$
in $\Z_{\bar{m}}^{*}$. Note that $d$ divides $\varphi(\bar{m}) =
n/e$.
The ideal $qR$ then factors into the product of $(re)$th powers of
$\varphi(\bar{m})/d = n/(de)$ distinct prime ideals $\frakp_{i}$, i.e.
$qR=\prod \frakp_{i}^{re}$.  Each prime ideal $\frakp_{i}$ has norm
$\abs{R/\frakp_{i}}=p^{d}$. The factorization of $qR$ for general
$q=p_1^{r_1}\ldots p_t^{r_t}$ then follows by recalling the unique
factorization of ideals for cyclotomic rings and that $qR=(p_1^{r_1}R)\ldots(p_t^{r_t}R)$.

\subsection{Gaussian Measures}
\label{sec:gaussian-measures}

For $s > 0$, the $n$-dimensional Gaussian function $\rho_{s}$ is
defined as \[\rho_s(\vecx) := \text{exp}(-\pi\length{\vecx}^2/s^2).\]

Normalizing this function gives the \emph{continuous} Gaussian
distribution $D_{s}$. More generally, we can define $D_{\matB}$ as the
distribution of $\matB\vecx$ where $\vecx$ is sampled from $D_{1}$. For
an invertible $\matB$, $D_{\matB}$ is proportional to
$\text{exp}(-\pi\vecx^{t}\matB^{-t}\matB^{-1}\vecx)$. For any
$\matB_1, \matB_2$, the sum of a sample from $D_{\matB_1}$ and
$D_{\matB_2}$ is distributed as
$D_{(\matB_1\matB_1^{t}+\matB_2\matB_2^{t})^{1/2}}$.

For an $n$-dimensional lattice $\Lat$ and a vector $\vecu \in \R^{n}$,
we define the \emph{discrete Gaussian distribution} $D_{\Lat+\vecu,s}$
as the discrete distribution with support on the coset $\Lat+\vecu$
whose probability mass function is proportional to $\rho_{s}$. 

We require the following now standard facts about Gaussian distributions
over lattices~\cite{DBLP:journals/siamcomp/MicciancioR07,DBLP:conf/stoc/GentryPV08,DBLP:conf/stoc/BrakerskiLPRS13}.
\begin{lemma}\label{lem:latt-facts}
Let $\lat$ be a lattice with associated basis $\matB$, and let
$\length{\tilde{\matB}}$ denote the length of the longest vector in
the Gram-Schmidt orthogonalization of $\matB$. Let $s \geq
  \length{\tilde{\matB}}\sqrt{\ln(2n(1+1/\epsilon))/\pi}$ for some $0 <
  \epsilon \leq 1/2$. Then 
\begin{enumerate}
\item There is a probabilistic polynomial-time algorithm that, given
$\vecc \in \R^{n}$ outputs a sample distributed according to
  $D_{\Lat+\vecc,s}$.
\item The distribution of $\vecx \bmod{\Lat}$, where $\vecx \gets
  D_{s}$, is within statistical distance $\epsilon/2$ of the uniform
  distribution over cosets of $\Lat$. 
\item Let $r > 0$. Then if we choose $\vecx \gets D_{r}$ and then
  choose $\vecy \gets D_{\Lat-\vecx,s}$, we have that $\vecx+\vecy$ is
  within statistical distance $8\epsilon$ of the discrete Gaussian $D_{\Lat,(r^2+s^2)^{1/2}}$.
\end{enumerate}
\end{lemma}