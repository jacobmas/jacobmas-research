\section{Preliminaries}
\label{sec:prelims}

\subsection{Notation}
\label{sec:notation}
Throughout this work, by ``ring'' we mean a commutative ring with
identity.  We identify the elements in $\Z_q$ with their coset
representatives in $[0,q)$. For an element $x \in \Z_q$, 
we define 
$\round{x}_{p} =\round{(p/q) \cdot x} \bmod{p}$,
where the latter $\round{\cdot}$ denotes standard
rounding\footnote{note that a number with fractional part .5 should
  always be rounded up} to the
nearest integer. For a ring element $y \in R_q$, the rounding
operation is performed coefficient-wise. We use $(a,b)$ to denote the $\gcd(a,b)$.

Note that we use $d$ and $w$ as matrix dimensional parameters instead
of $n$ and $m$ used in most $\lwe$ papers, because we reserve the
latter variables for the dimension and index of cyclotomic rings,
respectively.  We (sometimes implicitly) parameterize essentially all
variables asymptotically in terms of $\kappa$ instead of $n$, since we
use $n$ to denote the dimension of a ring and it may be small; in
particular, for $\Z$, we have $n=1$.
\subsection{Algebraic Number Theory Background}
\label{sec:algnum-theory}
\iflncs
We defer a review of most background
concepts from algebraic number theory to Appendix~\ref{sec:algebr-numb-theory}.
\else
Here we very briefly review concepts from algebraic number theory necessary
for (parts) of our work. 

For a thorough background on algebraic
number theory as it applies to cryptography based on ideal lattices,
see~\cite{DBLP:journals/jacm/LyubashevskyPR13,DBLP:conf/eurocrypt/LyubashevskyPR13},
and for proofs of various facts below one may consult, e.g.~\cite{samuel70}.


\paragraph{Cyclotomic Fields and Rings.}
For a positive integer $m$, the $m$th
\emph{cyclotomic number field} is $K=\Q(\zeta_{m})$, where $\zeta_m$ is some fixed arbitrary primitive $m$th root of unity (for each $m$,
we view it abstractly and not as any particular such root).  We denote
the ring of integers of $K$ as $\O_{K} = \Z[\zeta_{m}]$, and refer to
it as the $m$th cyclotomic ring. 

A cyclotomic number field
$K=\Q(\zeta)$ degree $n$ has exactly $n$ ring embeddings $\sigma_i: K
\to \C$, which can be defined by $\sigma_i(\zeta)=\zeta_i$ for $i \in
\Z_m^*$. The \emph{canonical embedding} is then defined as \[\sigma(x)
= (\sigma_1(x), \ldots, \sigma_{m-1}(x)) \in \C^{n}.\] We also define
the trace $\trace: K \to \Q$ and the (field) norm $N: K \to \Q$ as
$\trace(x)=\sum \sigma_j(x)$, $N(x)=\prod_{j}\sigma_j(x)$, and recall
that the trace is a ``universal'' $\Q$-linear function, in the sense
that  any linear function $L: K \to \Q$ can be expressed as
$L(a)=\trace(r\cdot a)$ for some fixed $r \in K$. 

The minimal polynomial of $\zeta_{m}$
over $\Q$ is the~$m$th \emph{cyclotomic polynomial}
$\Phi_{m}(X)=\prod_{i \in \Z_{m}^{*}} (X-\omega_{m}^{i}) \in \Z[X]$,
where $\omega_{m}=\exp(2\pi\sqrt{-1}/m) \in \C$ is the principal $m$th
complex root of unity, and the roots $\omega_{m}^{i} \in \C$ range
over all the \emph{primitive} complex $m$th roots of unity.
Therefore, $\O_{m}$ is a ring extension of degree $n=\varphi(m)$ over
$\Z$.  (In particular, $\O_{1}=\O_{2}=\Z$.)  Clearly, $\O_{m}$ is
isomorphic to the polynomial ring $\Z[X]/\Phi_{m}(X)$ by
identifying~$\zeta_{m}$ with~$X$, and has the ``power basis''
$\set{1,\zeta_{m}, \ldots, \zeta_{m}^{n-1}}$ as a $\Z$-basis. Note
that for $m\in \{1,2\}$, we have that $K=\Q$ and $R=\Z$.


\paragraph{Ideals.}\label{par:ideals} An (integral) \emph{ideal} $I$ of $R$ is an
additive subgroup of $R$ that is closed under multiplication by every
element of $R$. We denote the smallest ideal of $R$ containing the set
$S$ by $(S)$, and by $R/I$ the set of equivalence classes $x+I$ of $R$
modulo $I$. The norm of a (nonzero) ideal is the number of elements in
$R/I$.

A \emph{prime ideal} $I \subseteq R$ is such that if $ab \in I$,
then at least one of $a$ and $b$ is also in $I$. Over the ring of
integers of any algebraic number field, any ideal of $R$ can be
represented uniquely as a product of prime ideals.

\paragraph{Factorization of Ideals.}\label{par:ideal-fact} Let $q=p^r \in \Z$ be a prime power.  In the $m$th cyclotomic ring
$R=\O_{m} = \Z[\zeta_{m}]$ (which has degree $n=\varphi(m)$
over~$\Z$), the ideal $pR$ factors into prime ideals as follows. First
write $m = \bar{m}\cdot p^{k}$ where $p \nmid \bar{m}$. Let the
ramification index 
$e = \varphi(p^{k})$, and let the inertia degree~$d$ be the multiplicative order of $p$
in $\Z_{\bar{m}}^{*}$. Note that $d$ divides $\varphi(\bar{m}) =
n/e$.
The ideal $qR$ then factors into the product of $(re)$th powers of
$\varphi(\bar{m})/d = n/(de)$ distinct prime ideals $\frakp_{i}$, i.e.
$qR=\prod \frakp_{i}^{re}$.  Each prime ideal $\frakp_{i}$ has norm
$\abs{R/\frakp_{i}}=p^{d}$. The factorization of $qR$ for general
$q=p_1^{r_1}\ldots p_t^{r_t}$ then follows by recalling the unique
factorization of ideals for cyclotomic rings and that
$qR=(p_1^{r_1}R)\ldots(p_t^{r_t}R)$.
\fi

\paragraph{Modules.} A subset $M \subseteq K^d$ is an $R$-module if it
closed under addition and multiplication by elements of $R$. When $K$
is a number field and $R$ is its ring of integers, an $R$-module has a
so-called \emph{pseudo-basis} of linearly independent vectors $\vecb_i
\in K^d$ such that $M=\sum_{i=1}^{d}I_{i}\cdot \vecb_i$ for some
non-zero ideals $I_k$ of $R$. The representation of element of $M$
with respect to a pseudo-basis is unique, but two pseudo-bases can
generate the same module; however, these bases will always have the
same rank. 

\iflncs
\else
\subsection{Lattices, Ideal Lattices and Module Lattices}
\label{sec:module-lattices}
A lattice $\lat \subseteq R^n$ is a discrete additive subgroup
consisting of all integer linear combinations $\sum_{i \in
  m}x_i\vecb_i$, where $B=(\vecb_i)$ is called the basis.  


Under the canonical embedding, a fractional ideal $I$ with
$\Z$-basis $U=(u_1, \ldots, u_n)$ becomes a rank-$n$ \emph{ideal
  lattice} $\sigma(I)$ with basis $\{\sigma(u_1), \ldots,
\sigma(u_n)\} \subset H$. For convenience, we often identify the ideal
$I$ with its embedded lattice.

\fi


\iflncs

\else
\subsection{Gaussian Measures}
\label{sec:gaussian-measures}

For $s > 0$, the $n$-dimensional Gaussian function $\rho_{s}$ is
defined as \[\rho_s(\vecx) := \text{exp}(-\pi\length{\vecx}^2/s^2).\]

Normalizing this function gives the \emph{continuous} Gaussian
distribution $D_{s}$. More generally, we can define $D_{\matB}$ as the
distribution of $\matB\vecx$ where $\vecx$ is sampled from $D_{1}$. For
an invertible $\matB$, $D_{\matB}$ is proportional to
$\text{exp}(-\pi\vecx^{t}\matB^{-t}\matB^{-1}\vecx)$. For any
$\matB_1, \matB_2$, the sum of a sample from $D_{\matB_1}$ and
$D_{\matB_2}$ is distributed as
$D_{(\matB_1\matB_1^{t}+\matB_2\matB_2^{t})^{1/2}}$.

For an $n$-dimensional lattice $\Lat$ and a vector $\vecu \in \R^{n}$,
we define the \emph{discrete Gaussian distribution} $D_{\Lat+\vecu,s}$
as the discrete distribution with support on the coset $\Lat+\vecu$
whose probability mass function is proportional to $\rho_{s}$. 

We require the following now standard facts about Gaussian distributions
over lattices~\cite{DBLP:journals/siamcomp/MicciancioR07,DBLP:conf/stoc/GentryPV08,DBLP:conf/stoc/BrakerskiLPRS13}.
\begin{lemma}\label{lem:latt-facts}
Let $\lat$ be a lattice with associated basis $\matB$, and let
$\length{\tilde{\matB}}$ denote the length of the longest vector in
the Gram-Schmidt orthogonalization of $\matB$. Let $s \geq
  \length{\tilde{\matB}}\sqrt{\ln(2n(1+1/\epsilon))/\pi}$ for some $0 <
  \epsilon \leq 1/2$. Then 
\begin{enumerate}
\item There is a probabilistic polynomial-time algorithm that, given
$\vecc \in \R^{n}$ outputs a sample distributed according to
  $D_{\Lat+\vecc,s}$.
\item The distribution of $\vecx \bmod{\Lat}$, where $\vecx \gets
  D_{s}$, is within statistical distance $\epsilon/2$ of the uniform
  distribution over cosets of $\Lat$. 
\item Let $r > 0$. Then if we choose $\vecx \gets D_{r}$ and then
  choose $\vecy \gets D_{\Lat-\vecx,s}$, we have that $\vecx+\vecy$ is
  within statistical distance $8\epsilon$ of the discrete Gaussian $D_{\Lat,(r^2+s^2)^{1/2}}$.
\end{enumerate}
\end{lemma}

\fi

\subsection{Learning with Errors}
\label{sec:learning-with-errors}

Here we formally define the learning with errors
problem~\cite{DBLP:journals/jacm/Regev09} 
\iflncs
over modules.
\else
and its variants over rings and modules. 
As alluded to earlier, it is sufficient to simply define
module learning with errors ($\mlwe$) and module learning with
rounding ($\mlwr$), as the ring and general variants fall out as
special cases. 

\begin{definition}[$\mlwe$ Problem] Let $K$ be some finitely generated field extension of $\Q$, and denote
by $n$ its dimension. Let
$R=\O_{K}$ be its ring of integers, and let $\psi$ be some
distribution over $R$. Then we define the decision
version of $\mlwe_{R,d,w,q,\psi}$ as
follows.

The adversary gets
$(\matA \gets R_q^{d \times w}, \vecb \in R_q^w)$, and must
distinguish between the case where $\vecb = \matA^t\vecs+\vece \in R_q^w$, where
$\vecs \gets R_q^d$ uniformly at random and $\vece \gets
\psi^w$ and the case where $\vecb \gets R_q^w$ uniformly at
random.  

%When $\psi$ is a spherical Gaussian, we sometimes write
%$\mlwe_{R,d,w,q,\alpha}$ to denote $\mlwe_{R,d,w,q,D_{\alpha}}$, and
%similarly for $\lwe$ and $\rlwe$. 
\end{definition}

Under this definition, it is easy to see that $\mlwe$ is a
generalization of both $\rlwe$ and $\lwe$. When $d=1$, $\mlwe_{R,d,w,q,\psi}$ is equivalent to
$\rlwe_{R,w,q,\psi}$, while when $R=\Z$, $\mlwe_{R,d,w,q,\psi}$ is
equivalent to $\lwe_{d,w,q,\psi}$. 

\iflncs
\else
\paragraph{Hardness Guarantees.} We have quantum worst-case to average case reductions from $\sivp$ for
module lattices for $d \geq 1$, and we also have a (potentially
meaningful) classical reduction from $\gapsvp$ for $d\geq
2$~\cite{DBLP:journals/dcc/LangloisS15}. There are also many different
hardness reductions for the special cases of $\lwe$ and
$\rlwe$~\cite{DBLP:conf/stoc/Peikert09,DBLP:journals/jacm/Regev09,DBLP:journals/jacm/LyubashevskyPR13,DBLP:conf/stoc/BrakerskiLPRS13};
for details, consult the cited works.
\fi

\paragraph{Error Distributions.} For the purposes of this paper, we will
largely ignore the specific details of the error distribution, except for in
Section~\ref{sec:first-errorl-extend}, where we use Gaussian distributions of the form
described in Section~\ref{sec:gaussian-measures}.
For our main result, it is sufficient that the error distribution
is $B$-bounded, meaning that 
$\Pr_{\vecx \gets \psi}[\exists i\text{ such that}  x_i \notin [-B,B]] \leq \negl(\kappa)$.
\iflncs
\else
For completeness, we recall the following result of
Banszcyzyk~\cite{banaszczyk93:_new}, which allows us to upper-bound
the magnitude of the error coefficients in an $\lwe$ sample.

\begin{lemma}[Adaptation of ~\cite{banaszczyk93:_new} Lemma 1.5(i)] \label{lem:ban}
  For any lattice $\Lat$, $r>0$, and let $\vecx \gets D_{\Lat,r}$ we
  have that except with probability $2^{-2n}$,
  $\length{\vecx} \leq r\sqrt{n}$.
\end{lemma}

We caution that this result
cannot necessarily be used immediately as an upper-bound of (ring)
coefficients in $\rlwe$ and $\mlwe$  if one wishes to rely on
the existing worst-case reductions from hard ideal/module lattice problems. This is because we need bounds on the
magnitude of the individual coefficients of the error term, while the
error is sampled to be short with respect to the canonical
embedding. As in general (specifically, for cyclotomic polynomial rings
with an index that is the product of many distinct small
primes), the distortion between the coefficient embedding and the
canonical embedding may be superpolynomially
large in the index~\cite{erdos46:_coeff_of_cyclot_polyn}, these reductions may be
of less utility in such rings. However, in the rings of index $2^{k}$
(and more generally for rings of index $2^kp^{\ell}$, where $p$ is a
small odd prime), 
the distortion between the two embeddings will be very small, and so
relying on the worst-case reductions becomes meaningful. 
\fi
\subsection{Learning With Rounding}
\label{sec:learn-with-round}
The learning with rounding ($\lwr$) problem was introduced by
Banerjee, Peikert and Rosen as a ``derandomization'' of
$\lwe$~\cite{DBLP:conf/eurocrypt/BanerjeePR12}. 
As with learning with errors, we will simply define the module version
($\mlwr$), and note that $\rlwr$ and $\lwr$ fall out as special
cases. 

\begin{definition}[$\mlwr$ Problem]\label{def:mlwr} Let $K$ be an
  algebraic number field of dimension $n$,
$R=\O_{K}$ its ring of integers. Let $q\geq p \geq 1,d\geq 1 \in \Z$ be integers. Then we define the decision
version of $\mlwr_{R,d,w,q,p}$ as
follows:
The adversary gets
$(\matA \gets R_q^{d \times w}, \vecb \in R_p^w)$, and must
distinguish between two cases. In the first case, which we refer to as
the $\mlwr$ distribution,
\[\vecb = \round{\matA^t\vecs}_{p} = \round{(p/q)\matA^{t}\vecs} \in R_p^w,\] where
$\vecs \gets R_q^d$ uniformly at random,  while in the second
case, which we refer to use at the uniform case, $\vecb = \round{\vecu}_p$, where $\vecu \gets R_q^w$ uniformly
at random.
\end{definition}
Note that if $p$ divides $q$, the distribution of $\vecb$ is truly
uniform, while if $p \nmid q$, $\vecb$ will be slightly biased
towards certain values modulo $p$. 


%%% Local Variables: 
%%% mode: latex
%%% TeX-master: "latticereduction"
%%% End: 
