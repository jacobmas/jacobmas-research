\section{Reducing Extended-LWE to LWR}
\label{sec:redlwelwr}
\begin{theorem}\label{thm:extlwe-lwr}Let $\kappa$ be a security parameter on which all other parameters depend. Let $R$ be the ring of integers of an algebraic
  number field of dimension $n=O(\kappa)$. Let $\psi$ be an arbitrary
  coordinate-wise $B$-bounded distribution over $R$ for some $B
  >0$. Let $p, q, w=\poly(\kappa),d  \in \N$ such that $q \geq
  4eBwn\kappa p$. Let $\calZ \subseteq R^d$ be the set of vectors
  $\vecz$ such that $\trace(\inner{\vecz,\vece})$ extracts the $i$th
  coefficient of the $j$th ring element of $\vece$ for some $i \in
  [n], j \in [d]$.

If $\exists$ a probabilistic polynomial-time $\Adv$ succeeding with
advantage $\epsilon(\kappa) \geq (\kappa)^{-c}$ for some constant $c \geq 1$
 in distinguishing $\mlwr_{R,d,w,q,p}$ from
uniform, then there exists a probabilistic polynomial-time algorithm
$\Adv'$ succeeding with  advantage $\epsilon(nw)^{-c}/4 \geq (\kappa n
w)^{-c}/4$ in distinguishing
$\extmlwe_{R,d,w,q,\psi,\calZ,c}$ from uniform. 
\end{theorem}




\subsection{The Set $\setbad$}
\label{sec:defs-lemmas}
Before stating our reduction, we define and prove some basic results
about the set $\setbad_{B,q,p}$. This set, following the
notation of~\cite{DBLP:conf/eurocrypt/BanerjeePR12}, characterizes
those elements in $\Z_q$ that are within distance $B$ of an element
that rounds (with $\round{\cdot}_p$ as defined above) to a different
value than it does.

\begin{definition}\label{def:setbad} For  $B, p, q \geq 1$, we define $\setbad_{B,q,p} := \{x \in \Z_q: \round{x}_p \neq \round{x\pm
  B}_p\}.$
\end{definition}

The reason for the terminology is that, letting one of these elements
represent a coefficient with noise (from a $B$-bounded distribution)
in an $\mlwe$ sample, it may be the case that the noiseless version of
these coefficients rounds to a different value than the noisy one,
which is ``bad'' for our reduction.

We recall a bound on the probability that a uniform element in $\Z_q$ is in
$\setbad_{B,q,p}$.
\begin{lemma}{\cite[Lemma~2.7]{DBLP:conf/crypto/AlwenKPW13}}
\label{lem:sizebad} $\Pr_{x \gets \Z_q}[x \in \setbad_{B,q,p}] \leq 2Bp/q.$
\end{lemma}

This next lemma uses a standard Chernoff bound to upper bound the
probability any given number of elements in a uniformly $\vecb \in
R_q^{w}$ will be in $\setbad$.

\begin{lemma}
\label{lem:badbound}
Let $B,q,p,\kappa,n,w$ satisfy the conditions in
Theorem~\ref{thm:extlwe-lwr}.  Let $\vecb \gets R_q^{w}$ uniformly at
random, and let $S = \{b_{i,j} \in \setbad_{B,q,p}\}$.  Then for all
$c \geq 1$,
\[\Pr[\abs{S} \geq c] \leq \kappa^{-c}/2\]
\end{lemma}
\begin{proof}
  Let $X_i$ be the event that $b_i \in \setbad_{B,q,p}$.  Let $X =
  \sum_{i \in [m]}X_i, \mu = E[X]$. First, by linearity of expectation
  and Lemma~\ref{lem:sizebad}, we have that $\mu \leq 2Bnwp/q \leq
  1/(2e\kappa)$.

Let $\delta = \tfrac{c}{\mu}-1 > 0$. Then
\[\Pr[X > c] <
\left(\frac{e^{\delta}}{(1+\delta)^{1+\delta}}\right)^{\mu}
= \frac{e^{\tfrac{\delta c}{1+\delta}}}{(1+\delta)^{c}} \leq
\left(\frac{e}{1+\delta}\right)^c = \kappa^{-c}/(2c)^{c} \leq \kappa^{-c}/2.\]
\end{proof}
\subsection{Reduction}
\label{sec:reduction}

%Let $c=k+1$. 
We begin by choosing which $c$ error elements we will receive as
hints, by choosing uniformly at random $c$ distinct
elements 
\[\setguesses := \{\ell_1, \ell_2, \ldots, \ell_c \gets [w] \times [n]\},\]
where the $\ell_{\iota}=(i_{\iota},j_{\iota})$ are tuples representing
the $j$th coefficient of the $i$th element in the error vector $\vece
\in R_q^{w}$.

Since the trace is a ``universal'' $\Q$-linear function, there exist
(efficiently computable) vectors $\vecz_{\ell_1}, \vecz_{\ell_2},
\ldots, \vecz_{\ell_c}$ such that
$\trace(\inner{\vece,\vecz_{\iota}})=\vece_{\ell_{\iota}}$ for all $i$.
We query the $\extmlwe_{R,d,w,q,\psi,\calZ,c}$ oracle with these
vectors, receiving back $(\matA \in R_q^{d \times w}, \vecb \in
R_q^w,(e_{\ell_1}, \ldots, e_{\ell_c})).$

Let \[T=\{(i,j) : b_{i,j} \in \setbad_{B,q,p}\},\] where
$b_{i,j}$ refers to the $j$th coefficient of the $i$th ring element.
We have the following abort events.
\begin{description}
\item[E.1] If $\abs{T} > c$, we abort.
\item[E.2] Otherwise, if $T \not \subseteq \setguesses$, we abort.
\item[E.3] Otherwise, we abort with
probability 1-$1/\binom{wn-\abs{T}}{c-\abs{T}}$.  
\end{description}
Now, if we have not aborted, set $\tilde{\vece}_{i,j}=e_{i,j}$ if $(i,j) \in
\setguesses$, $\tilde{\vece}_{i,j}=0$.  Let $\vecb' :=
\round{\vecb - \tilde{\vece}}_p$, where the rounding is done
element-wise, and let $y \gets \Adv(\matA, \vecb')$. $\Adv'$ outputs
$y$ as its guess (where 0 corresponds to $\extmlwe/\mlwr$,
respectively, 1 to uniform).

%\end{proof}


\subsection{Analysis}
\label{sec:analysis}

Here we prove correctness of the above reduction. To do so, we need to
prove two things. First, we need to show that we invoke $\Adv$
(i.e. we do not abort and output a random bit) with non-negligible
probability. Second, conditioned on not aborting, we need to show that
the distribution of the output sent to $\Adv$ is within statistical
distance at most $\epsilon/2$ of the true $\mlwr$ distribution if the
oracle samples $\vecb$ from $\extmlwe$, and within distance at most
$\epsilon/2$ of the uniform distribution if the oracles samples
$\vecb$ uniformly and independently.

The following simple lemma shows that if the distribution of $\abs{T}$
differed depending on whether the oracle gives samples from $\extmlwe$
or uniform, then we would have an (alternative standalone) attack on
$\extmlwe$. As a result, we need only consider the distribution of
$\abs{T}$, when the oracle outputs a truly uniform $\vecb$, where it
is simpler to analyze.
\begin{lemma}\label{lem:t_ind} Under the $\extmlwe_{R,d,w,q,\psi,\calZ,c}$ assumption,
  we have that 
\[\abs{\Pr[\abs{T} > c \mid \text{oracle is \extmlwe}] - \Pr[\abs{T}
    > c \mid \text{oracle is uniform}]} \leq \negl(n)\]
\end{lemma}
\begin{proof}
  If there was a non-negligible difference in the two probabilities,
  we could simply query the oracle and compute $\abs{T}$ to give an
  attack on $\extmlwe_{R,d,w,q,\psi,\calZ,c}$ succeeding with non-negligible
  probability, without having to rely on $\Adv$.\end{proof}


We can now upper bound the probability of abort event E.1.
\begin{lemma}\label{lem:boundE1}
\[\Pr[\mathbf{E.1} \mid \text{oracle is }\extmlwe] \leq \negl(\kappa) + \Pr[\mathbf{E.1} \mid \text{oracle is uniform}] \leq \epsilon/2\]
\end{lemma}
\begin{proof}
The first inequality is a direct consequence of Lemma~\ref{lem:t_ind}, while the second follows immediately from~\ref{lem:badbound}.
\end{proof}


The next lemma shows that the distribution conditioned only on $\textbf{E.1}$ not happening is exactly the same as the distribution conditioned
on none of the abort events happening, and that the probability of none of the abort events happening is at least $(1-\epsilon/2)/\tbinom{nw}{c}$.

\begin{lemma}
\label{lem:abortequiv}
\[\Pr[\neg (\mathbf{E.1} \vee \mathbf{E.2} \vee \mathbf{E.3})] \geq \frac{(1-\epsilon/2)}{\tbinom{nw}{c}}.\]
Moreover, for any $\matA^* \in R_q^{d \times w},\vecb^* \in R_q^w \in \Z_q^c$, we have 
\[\Pr[(\matA,\vecb)=(\matA^*,\vecb^*) \mid \neg \mathbf{E.1}] = \Pr[(\matA,\vecb)=(\matA^*,\vecb^*) \mid \neg (\mathbf{E.1} \vee \mathbf{E.2} \vee \mathbf{E.3})]  \]
\end{lemma}
\begin{proof}
  First, we consider $\mathbf{E.2}$. Consider any fixed possible value
  $t$ for the number of elements in $\setbad$.  For this fixed value of $\setbad$, there are
  $\binom{nw-t}{c-t}$ possible distinct sets $\setguesses$ of indices
  that cover all elements in $\setbad$, while there are $\binom{nw}{c}$
  total possible distinct sets  $\setguesses$ of indices.

Since the elements of $\setguesses$ are chosen uniformly and
independent of everything else, we have that 

\[\Pr[\neg \mathbf{E.2} \mid  \neg \mathbf{E.1} \wedge \abs{\setbad}=t] =
\tbinom{wn-t}{c-t}/\tbinom{wn}{c}\]

Next, by the definition of $\mathbf{E.3}$,
\[\Pr[\neg \mathbf{E.3} \mid \neg (\mathbf{E.1} \vee \mathbf{E.2}) \wedge
  \abs{\setbad}=t] = 1/\tbinom{wn-t}{c-t}\]


\[\Pr[\neg \mathbf{E.3} \wedge \neg \mathbf{E.2} \mid \neg \mathbf{E.1} \wedge
\abs{\setbad}=t] = 1/\tbinom{wn}{c}\]
independently of $\abs{\setbad}$ and thus independently of the value of  $\matA, \vecb$ received from the
$\extmlwe$ oracle. 
\[\Pr[\neg \mathbf{E.3} \wedge \neg \mathbf{E.2} \mid \neg \mathbf{E.1}] =
1/\tbinom{nw}{c}.\]
Combining the above result with Lemma~\ref{lem:boundE1} gives
\[\Pr[\neg (\mathbf{E.1} \vee \mathbf{E.2} \vee \mathbf{E3}] \geq
\frac{1-\epsilon/2}{\binom{nw}{c}}\]
\end{proof}

Finally, we show that conditioned on not aborting, the samples sent to $\Adv$ are distributed sufficiently close to the desired distributions. 
\begin{lemma}\label{lem:round-correct}
  The distribution of $(\matA, \vecb')$ sent to $\Adv$ is within
  statistical distance at most $\epsilon/2$ of the true $\mlwr$
  distribution if the oracle sampled $\vecb$ from $\extmlwe$, and
  uniform if it sampled it uniformly at random.\end{lemma}
\begin{proof}
  Since the reduction only ever
  sends $(\matA,\vecb')$ if none of the abort events (in particular,
  $\mathbf{E.2}$) happen, we receive as our $\extmlwe$ hints the error
  coefficients for every $b_{i,j} \in \setbad_{B,q,p}$.  Let
  $\vecy=\vecb - \tilde{\vece} \in R_q^{w}$ (with $\tilde{\vece}$ as
  defined above). 

  In the case the oracle samples $\vecb=\matA^{t}\vecs+\vece$ from
  $\extmlwe$, we have that every coefficient of every ring element is
  either errorless or it is ``good'', meaning it is not in
  $\setbad$. As a result, we have that
\[\round{\vecy}_p=\round{\matA^{t}\vecs}_p,\]
so, combining with Lemmas ~\ref{lem:boundE1} and
~\ref{lem:abortequiv} , we have that it is distributed within distance
$\epsilon/2$ of $\mlwr$.

In the case the oracle samples $\vecb$ uniformly at random and
independently, we have that $\tilde{\vecx}$ is independent of $\vecb$,
and so $\vecy$ remains truly uniform, making $\round{\vecy}_p$ have
the identical distribution to the ``uniform'' distribution.
\end{proof}

By Lemma~\ref{lem:round-correct} and our assumption on $\Adv$,
conditioned on none of the abort events happening, $\Adv$ must succeed
in distinguishing with advantage at least $\epsilon/2$.

Combining this with Lemma~\ref{lem:abortequiv} gives

\[\epsilon' \geq \frac{\epsilon/2(1-\epsilon/2)}{\binom{nw}{c}} \geq \epsilon(nw)^{-c}/4,\]
as required.\medskip


This completes the reduction's correctness proof.





%%% Local Variables: 
%%% mode: latex
%%% TeX-master: "latticereduction"
%%% End: 
