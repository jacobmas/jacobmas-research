\section{Extended-LWE}
\newcommand{\ExptUnif}{\algo{ExptUnif}}
\newcommand{\ExptLWE}{\algo{ExptLWE}}

\label{sec:ext-lwe}
In this section we define $\extlwe$ and its variants, and give an
improved reduction from $\lwe$ to $\extlwe$\iflncs. We defer a
reduction from $\mlwe$ to $\extmlwe$ to the full version due to space considerations. \else, and a
reduction from $\mlwe$ to $\extmlwe$.\fi

\subsection{Definitions}
\label{sec:definitions}

We provide a definition of the problem in the context of module
lattices ($\extmlwe$), following the definition of Brakerski et
al. for regular $\extlwe$~\cite{DBLP:conf/stoc/BrakerskiLPRS13}.

\begin{definition}\label{def:extmlwe}For a number field $K$ with ring of integers $R$,
  dimensional parameters $d, w \in \Z$, modulus $q \geq 1$, a set
  $\calZ \subseteq K^{d}$, a parameter $k$ for the number of
  hints, and a distribution $\psi$ over $R$, the
  $\extmlwe_{R,d,w,q,\psi,k,\calZ}$ problem is defined as follows: The
  challenger gets to choose $\vecz_1, \ldots, \vecz_{k} \in \calZ$ and
  receives back a tuple
$(\matA, \vecb, (\trace(\inner{\vece,\vecz_{i}}))_{i \in [k]}) \in R_q^{d
  \times w} \times R_q^{w} \times \Q_q^k$.

The goal is to distinguish between two cases. In the first, which we
refer to as the $\extmlwe$ distribution, $\matA$ is
chosen uniformly at random, $\vece \gets \psi^{w}$, and $\vecb =
\matA^{t}\vecs+\vece \pmod{q}$ for $\vecs$ chosen uniformly at
random. The second case, which we refer to as the uniform distribution, is identical, except that $\vecb$ is chosen
uniformly at random and independently of everything else.
When $\psi=D_{\alpha}$, we sometimes write
$\extmlwe_{R.d,w,q,\alpha,k,\calZ}$ to mean
$\extmlwe_{R.d,w,q,D_{\alpha},\calZ}$, and when the ring $R$ and
$\calZ$ are clear in context, we often omit them.
\end{definition}

\iflncs In Section~\ref{sec:redlwelwr}, $\calZ$ will be the set of
(scaled down by the dimension of $K$) columns of the identity matrix
$\matI_{w}$. For a more thorough definition, see Section~\ref{sec:set-quality}.
\else
Following Brakerski, et al, we define the \emph{quality} of a set
$\calZ \subseteq \Z^{m}$.

\begin{definition}\label{def:quality} For a real $\xi > 0$ and a set $\calZ \subseteq
  K^w$,
  we say that $\calZ$ is of \textnormal{quality} $(\xi,t)$ if given
  any $\vecz_1, \ldots, \vecz_t \in \calZ$, we can efficiently
  construct a unimodular matrix $\matU \in \Z^{w \times w}$ such that,
  if $\matU' \in \Z^{w \times (w-t)}$ is the matrix obtained from
  $\matU$ be removing its $t$ leftmost columns, then all the columns
  of $\matU'$ are orthogonal to $\text{span}(\vecz_1,\ldots,\vecz_t)$
  and its largest singular value is at most $\xi$.
\end{definition}

For our reduction in Section~\ref{sec:redlwelwr}, we will have $\calZ$
be the set of (scaled down by the dimension of $K$) columns of the
identity matrix $\matI_{w}$. We can easily see that $\calZ$ is of
quality $(1,t)$ for all $t < w$ by letting $\matU$ be a suitable
permutation of the rows of the identity matrix.
\fi


%
\subsection{Miscellaneous Lemmas}
\label{sec:miscellaneous-lemmas}

\begin{lemma}\label{lem:collision_lemma}
Let $\chi$ be a distribution with support on $0, 1, \ldots, n-1$ for some $n$. Then
\[\Pr_{x, x' \gets \chi}[x=x']\geq 1/n\]
\end{lemma}
\begin{proof}
Our proof proceeds by induction on $n$. For the case $n=1$ the only possible distribution for $\chi$ is the one that outputs $0$ with probability $1$, which covers the base case.

Now, assume the lemma is true for all distributions with support on $0, 1, \ldots, k-1$ for any $k < n$. Now, consider an arbitrary distribution $\chi$ with support on $0,1,\ldots, n-1$. We have that 
\begin{align*}
\Pr_{x,x' \gets \chi}[x=x']&=\sum_{i=0}^{n-1}\Pr_{x, x' \gets \chi}[x=i \wedge x' = i]\\
&=\sum_{i=0}^{n-1}\Pr_{x \gets \chi}[x=i]\Pr_{x' \gets \chi}[x'=i]\\
&=\sum_{i=0}^{n-1}(\Pr_{x \gets \chi}[x=i])^2\\
\end{align*}
\end{proof}

%%% Local Variables: 
%%% mode: latex
%%% TeX-master: "latticereduction"
%%% End: 

\subsection{Reduction from LWE for Arbitrary  Moduli}
\label{sec:impr-reduct-over}

Here we give an improved dimension-preserving reduction from LWE to
extended LWE. This reduction can handle any (polynomially-sized)
modulus, while the previous reduction by Alperin-Sheriff and Peikert required that all prime factors
of the modulus be (with overwhelming probability) larger by a factor
of 2 than the inner product of the hint vector with the error
vector~\cite{DBLP:conf/pkc/Alperin-SheriffP12}.

To aid clarity, we follow the previous work and give our reduction in
terms of the knapsack forms of $\lwe$ and $\extlwe$. We recall that a
(tight) equivalence between $\lwe$ and its knapsack form applies for
$m \geq n + \omega(\log{n})$. For a proof,
consult~\cite[Lemmas 4.8 and 4.9]{DBLP:conf/crypto/MicciancioM11}.


To prove our result, we define a hybrid variant of $\extlwe$, which we
denote as $\extlwe^{d}$. Our definition is somewhat similar
to the hybrid variant of $\lwe$ appearing in the proofs of previous
search-to-decision $\lwe$ reductions,
e.g.~\cite{DBLP:conf/eurocrypt/MicciancioP12}. In this form, one is
given $\matH \in \Z_q^{(m-n) \times m}$, $\vecc \in \Z_q^{m-n}$, and
one must determine whether $\vecc$ is uniformly or random or equal to
$\vecc=\matH\vecx$ for $x \gets \chi^m$. 


Upon
receiving $\vecz$, $\extlwe_{n,m,q,\chi}^{d}$ outputs $(\matH \gets \Z_q^{(m) \times (m+n)}, \vecc=\matH\vecx+d\vecv
\bmod{q}, \vecz, \inner{\vecx,\vecz})$,
where $\vecx \gets \chi^{m+n}$ and $\vecv \gets \Z_q^{m}$ uniformly at
random. We have that $\extlwe_{n,m,q,\chi}^{q}=\extlwe$ and
$\extlwe_{n,m,q,\chi}^1=U$ is the uniform distribution.

First we prove a lemma which establishes that a sufficiently large
advantage in distinguishing $\extlwe_q^d$ from uniform (where sufficiently large is
both absolute and relative to the advantage for smaller factors) must exist for
some factor $d$ of $q$. The form of the advantage here is chosen to
make our proof go through. 
 
\begin{lemma}\label{lem:general_mod}
  Let $\Adv$ be an algorithm with advantage $\epsilon>0$ in
  distinguishing between $\extlwe_q$ and uniform for some
  $q=\poly(\lambda)$. Then for any constant $k \geq 0$, $\exists$ a minimal
  factor $d$ of $q$ such that the advantage of $\Adv$ in
  distinguishing $\extlwe_q^{1}$ and $\extlwe_q^{d}$ is at least
  $\epsilon/(q/d)^k$, and it can be found efficiently (in
  $\poly(\lambda)$ time).
\end{lemma}

\begin{proof}
  By hypothesis, $\Adv$ has advantage $\epsilon$ in distinguishing
  $\extlwe_q^{1}$ and $\extlwe_q^{q}$, so $d=q$ satisfies the
  conditions of lemma, and therefore by the well-ordered principle a
  minimal such $d>1$ must exist (that $d$ must be greater than $1$ is
  obvious since at $1$ the distributions are identical). Moreover, we
  can find the minimal $d$ efficiently by testing the behavior of
  $\Adv$ on input from $\extlwe_{q}^{d}$ for every factor $d$. Note
  that the factors $d$ of $q$ can be found efficiently via trial
  division since $q$ itself is polynomially sized.
\end{proof}

The following two lemmas will allow us to bound the probability of
each hybrid $\extlwe^{d}$ occurring in our main proof.

\begin{lemma}\label{lem:monotonicity}Let $q>1$ be an arbitrary integer,
  $\chi$ be a discrete Gaussian distribution (with arbitrary parameter).
Then for all $f, d$ such that $f \mid d \mid q$, 
\[\Pr_{t \gets \chi}[\gcd(t,q)=f \text{ and } t \neq 0] \geq \varphi(d/f)\Pr_{t
  \gets \chi}[d \mid \gcd(t,q) \text{ and } t \neq 0]\]
\end{lemma}

\begin{proof}
Noting that a Gaussian distribution
is always monotonically decreasing as a function of the \emph{absolute
value} of $x$, the inequality follows immediately from the definition of the Euler phi function.\end{proof}

\begin{corollary}\label{cor:prob_val}
Let $q > 1$ be arbitrary, $\chi$ be a discrete Gaussian distribution
with arbitrary parameter. Then for all $f \mid d \mid q$,  $1 \leq f <
d$, $\Pr_{t \gets \chi}[f = \gcd(t,d)] \leq \tfrac{1}{f}$.
\end{corollary}
\begin{proof}
Using Lemma~\ref{lem:monotonicity} for the third inequality and the
property that $n=\sum_{d \mid n} \varphi(n/d)$ for the fourth equality   we have that 1
\begin{align*}
\geq \sum_{x \mid d}\Pr_{t \gets \chi}[(t,d)=x, t
    \neq 0]&\geq \sum_{x \mid f}\Pr_{t \gets \chi}[(t,d)=x, t \neq
  0]\\
\geq \sum_{x \mid f} \varphi(f/x)\Pr_{t \gets \chi}[f \mid (t,d),  t
  \neq 0]&=f\Pr_{t \gets \chi}[f \mid (t,d), t \neq 0]&\geq f\Pr_{t \gets \chi}[f = (t,d)]\\
\end{align*}
\end{proof}
\begin{lemma}\label{lem:prob_d}
Let $\chi$ be a probability distribution over the integers, symmetric
about the origin, such that the probability of $x$ decreases
monotonically as a function of the \emph{absolute value} of $x$. Then
$\Pr_{t \gets \chi}[\gcd(t,d)=d] \geq \tfrac{1}{2d-1}$.
\end{lemma}
\begin{proof}
We have 
\begin{align*}
\Pr_{t \gets
  \chi}[\gcd(t,d)=d]&=\frac{\Pr[t=0]+2\sum_{i=1}^{\infty}\Pr[t=di]}{\Pr[t=0]+2\sum_{i=1}^{\infty}\sum_{j=0}^{d-1}\Pr[t=di+j]}\\
&\geq
  \frac{\Pr[t=0]+2\sum_{i=1}^{\infty}\Pr[t=di]}{(2d-1)\Pr[t=0]+2d\sum{i=1}^{\infty}\Pr[t=di]}
  & 
\geq \frac{1}{2d-1}.
\end{align*}
\end{proof}

\begin{theorem}\label{thm:main_new_extlwe}Let $\chi=D_{\Z,s}$ be a
  discrete Gaussian with parameter $s$. Let $q=\poly(\lambda)$ for
  some parameter $\lambda$.
  $\Adv$ be an algorithm succeeding in time $t$ with advantage $\epsilon$ in
  distinguishing $\extlwe_{n,m,q,\chi}$ and uniform. Then there exists an algorithm
  $\Adv'$ succeeding in time $t+\poly(\lambda)$ with advantage at
  least $\epsilon/q^4$ in distinguishing $\lwe_{n,m,q,\chi}$ from uniform.
\end{theorem}
\begin{proof}
  By Lemma~\ref{lem:general_mod}, for any given constant $k$ (to be optimized below) we can efficiently find the minimal $d$ such
  that $\Adv$ distinguishes between $\extlwe_{n,m,q,\chi}^{1}$ and
  $\extlwe_{n,m,q,\chi}^{d}$ with advantage at least
  $\epsilon/(q/d)^k$. 

  We show how to use $\Adv$ to distinguish between $\lwe$ and
  uniform.  We receive a matrix $\matH \in \Z_q^{m \times (m+n)}$
  and a vector $\vecc \in \Z_q^{m}$  We then
  receive $\vecz \in \Z_q^{m+n}$ from $\Adv$ and sample
  $\vecx' \gets \chi^{m+n}$, $\vecv \gets \Z_q^{m}$,
  $\vecw \gets \Z_q^{m}$ and let $\matH' = \matH - \vecv\vecz^t,\quad \vecc' = \vecc -
\vecv\inner{\vecz,\vecx'}+d\vecw$. We then run $\Adv$ on $(\matH',
\vecc',\vecz,\inner{\vecx',\vecz})$. If $\Adv$ outputs accept, we output
$\lwe$, and if $\Adv$ outputs reject, we output uniform.


We now analyze our success probability. Let $\delta$ be the
probability that $\Adv$ accepts on $\extlwe^{1}$.  Without loss of
generality (by flipping the output bit of $\Adv$ above, if necessary) $\Adv$ accepts
on $\extlwe^{d}$ with probability at least $\delta+\epsilon/(q/d)^k$.

First, consider the case that our input was from the uniform distribution, then it is easy
to see that our output to $\Adv$ is always $\extlwe^{1}$. (the uniform
distribution).


Now, consider the case that our input was from $\lwe$, so $\vecc=\matH\vecx$. Then our output
to $\Adv$ is 
$\matH'=\matH-\vecv\vecz^t,\quad \vecc'=\matH'\vecx+\vecv\inner{\vecz,\vecx-\vecx'}+d\vecw$.
Let
$d'=\inner{\vecz,\vecx-\vecx'}=\inner{(\vecz,\vecz),(\vecx,-\vecx')}$,
and note that $d'$ is distributed according to $D_{\Z,s\sqrt{2}\length{\vecz}}$.
Now,  let $g=\gcd(d,d')$, so
$gt=d, gt'=d$ for some coprime $t,t'$. The
Chinese Remainder Theorem gives that $\vecy=t\vecv+t'\vecw$ is uniform $\bmod{q/g}$,
so $\vecc'=\matH'\vecx+g\vecy$ for a uniform $\vecy$, and our output to $\Adv$ is
distributed exactly according to
$\extlwe^{g}$. 

By the condition on the minimality of $d$, for all $g \mid d$,
$\Adv's$ advantage in distinguishing $\extlwe^{g}$ and
$\extlwe^1$ is at most $\epsilon/(q/g)^k$, so it must be the case that
$\Adv$ accepts with probability \emph{at least}
$\delta-\epsilon/((q/g)^k)$ on input $\extlwe^{g}$.


Let $E$ be the event that $\Adv$ accepts on our output to it, given
that our input is $\lwe$. 
We have that
\begin{align*}
\Pr[E]-\delta&=(\sum_{f\mid d}\Pr[g=f]\Pr[E \mid g=f]) - \delta &\geq \tfrac{\epsilon}{q^k}\tfrac{d^k}{(2d-1)} - \sum_{1 < f<d}\Pr[g=f]f^k\\
&\geq \tfrac{\epsilon}{q^k}(\tfrac{d^k}{2d-1} - \sum_{1<f<d, f \mid
  d}f^{k-1})
&\geq \tfrac{\epsilon d^{k-1}}{q^k}(\tfrac{d}{2d-1} - \sum_{1<f<d, f \mid
  d}\tfrac{1}{f^{k-1}})\\
&\geq  \tfrac{\epsilon
  d^{k-1}}{q^k}(\tfrac{d}{2d-1}-\sum_{f=2}^{\infty}\tfrac{1}{f^{k-1}})
&= \tfrac{\epsilon d^{k-1}}{q^{k}}(\tfrac{d}{2d-1} - (\zeta(k-1)-1)),
\end{align*}
where the second inequality follows from Lemma~\ref{lem:prob_d} and our
condition on $d$, the third inequality follows from Corollary~\ref{cor:prob_val}
 and the last equality follows from the definition of the Riemann zeta
 function.

The optimal value of $k$ is such that $\zeta(k-1)-1$ is just a
bit smaller than $0.5$, say $k=3.2$. We then have that $\min_{d \geq 2 \in \Z}(d^{3.2}(1/(2d-1)-(\zeta(2.2)-1)) \geq .8$.

As a result, for our choice of $k=3.2$,
$\Pr[E] \geq \delta+.8\epsilon/q^{3.2}$, and so we can distinguish
$\lwe$ and uniform with advantage at least $.8\epsilon/q^{3.2}$.
\end{proof}
\iflncs
\else
We note that bounding the sum with the zeta function above results in
a rather loose bound and that significantly tighter bounds can be
achieved for most moduli. 
\fi
\iflncs
\else
\subsection{Extended-LWE Over Module Lattices}
\label{sec:extended-lwe-over}


Here we show that an analogous version of the $\extlwe$ reduction
found in~\cite{DBLP:conf/stoc/BrakerskiLPRS13} holds over module
lattices as well. In order to prove this theorem, we follow the
path of Brakerski et al.~\cite{DBLP:conf/stoc/BrakerskiLPRS13}.  We
first prove the reduction to the intermediate problem
``first-is-errorless-M-LWE'' and then proceed to prove the reduction
from ``first-is-errorless-M-LWE.'' to $\extmlwe$.


\subsubsection{First-Is-Errorless-LWE}
\label{sec:first-errorless-lwe}

As in the previous work, we have the intermediate step of
``first-is-errorless'' $\mlwe$, in which the first inner product
contains no error. The proof for this general case essentially follows
that of Brakerski et al., with a few important differences stemming
from the form of $R$ will not in general be a Euclidean domain or
even a principal ideal domain. This requires some careful proof
restructuring.
\begin{definition}\label{def:first-errorless-mlwe}
  Let $K$ be an algebraic number of field, $R=\O_{K}$ be its ring of
  integers. Let $q>1,d\geq 2$ be integers, and let $\psi$ be a
  distribution over $R$. Then the
  ``first-is-errorless'' version of $\mlwe$ requires the adversary to
  distinguish between two cases. In the first case, the adversary
  receives $\matA \in R_q^{d \times w},\vecb = \matA^t\vecs+\vece \in
  R_q^w$, where $\vecs \gets R_q^d$ uniformly at
  random and $\vece \gets (0, \vece')$, where $\vece' \gets
  \psi^{w-1}$.  In the second case, $\vecb \gets R_q^m$ uniformly at random.
\end{definition}

We first prove a brief lemma relating to the probability that the
greatest common divisor of $qR$ and the 
principal ideals generated  by $d$
elements chosen uniformly at random modulo $R_q$  is in fact all of
$R$ (making all of these ideals coprime). 
\begin{lemma}\label{lem:prob-relprime}
Let $q=\prod_{i \in
    [\ell]}p_i^{e_i}$. Let $R=\calO_{m}$ and write $m=\bar{m}\cdot
  p_i^{\kappa_i}$, where $p \nmid \bar{m}$. Let $f_i$ be the
  multiplicative order of $p_i$ modulo $\bar{m}$. Let $a_1, \ldots,
  a_d \gets R_q$ uniformly at random (strictly speaking, elements in
  $R$ from
  some canonical set of representatives of $R_q$). Then we have that 
$R=\gcd(qR,\inner{a_1},\ldots,\inner{a_d})=qR+\inner{a_1}+\ldots+\inner{a_d}$,
except with probability at most
\begin{equation}
\label{eq:upp-bound-lin-ind}\sum_{i \in
  \ell}\left(\frac{\varphi(\bar{m})}{f_i}p_i^{-d\cdot f_i} \right)
\end{equation}
\end{lemma}
\begin{proof}
  We have that the sum of all the ideals above will be equal to $R$
  unless $a_1, \ldots, a_d$ all lie in some prime ideal $\frakp_i$ of
  $qR$.  Recall from~\ref{par:ideal-fact} that a prime ideal
  $\frakp_{i,j}$ dividing $p_iR$ has norm
  $\length{R/\frakp_{i,j}}=p_i^{f_i}$, the probability of this event
  is at most $p_i^{-d\cdot f_i}$. Equation~\ref{eq:upp-bound-lin-ind}
  then follows from a simple union bound.
\end{proof}


We now prove the reduction.
\begin{lemma}\label{lem:first-errorless}
  Let $d \geq 2$, $w$, $q > 1$, $R=\calO_{m}$, $\psi$ be an error distribution
  over $R$. There is an efficient (transformation) reduction
  from $\mlwe_{R,d-1,w-1,q,\psi}$ to the first-is-errorless variant of
  $\mlwe_{R,d,w,q,\psi}$ that reduces the advantage by at most
$\sum_{i \in
  \ell}\left(\frac{\varphi(\bar{m})}{f_i}p_i^{-d\cdot f_i} \right)$, where
  $f_i$ and $\bar{m}$ are as defined in Lemma ~\ref{lem:prob-relprime}.
\end{lemma}

\begin{proof}
  The reduction first chooses $\veca'\gets R_q^d$. 
If we have that $qR+\inner{a'_1}+\ldots+\inner{a'_d} \subset R$, the reduction
aborts. Otherwise, it finds, via a generalization of the
  Euclidean algorithm~\cite{extendeuclidideals}, a $d \times d$
  invertible matrix  $\matR \in R_q^{d \times d}$ such that
  $\matR\veca'=(1,0,\ldots,0)^t$, and sets $\matU = \matR^{-1}$.  

It also picks a uniform element $s_0 \in
R_q$. It then outputs $(\veca', s_0)$  as the first sample. For
each of the $w-1$ remaining samples, it first samples $(\veca,b)$ from
the given $\mlwe$ oracle, picks a fresh uniformly random $d \in
R_q$, and outputs $(\matU(d|\veca), b+ (s_0\cdot d)$, where the
vertical bar denotes concatenation. 

Conditioned on not aborting, we will verify that the
output is distributed according $\mlwe_{R,d,m,q,\psi}$. Combining this
with the bound in Lemma~\ref{lem:prob-relprime} will give our desired result.

We now verify correctness, conditioned on not aborting. First, we consider the
case that
$(\veca,b)$ samples are uniform and independent. In this case it is
easy to see that the outputs remain uniform, since $(d|\veca)$ is
uniform, $\matU$ is invertible, and each $b$ is uniform and independent of
everything else. Secondly, if the samples are from $\mlwe$ with
respect to a secret $\vecs$, it is easy to verify that the outputs are
distributed according to the first-is-errorless variant
$\mlwe_{R,d,w,q,\psi}$ with secret $\vecs'=\matR^{t}(s_0|\vecs)
\bmod{q}$. Since $\matU$ is invertible modulo $q$ and $(s_0|\vecs)$ is
uniform, so is $\vecs'$, which proves correctness.
\end{proof}

By iterating this reduction and applying a union bound, we immediately have the following
corollary which gives a reduction to a first-$t$-are-errorless variant
of $\mlwe$. We use this corollary in the next section to reduce to
$\extmlwe$ with $t$ hints.
\begin{corollary}\label{lem:first-t-errorless}
  Let $d > t \geq 0$, $w$, $q > 1$,  $R=\calO_{m}$, $\psi$ be an error
  distribution over $R$. There is an efficient (transformation) reduction from
  $\mlwe_{R,d-t,w-t,q,\psi}$ to the first-$t$-are-errorless variant of
  $\mlwe_{R,d,w,q,\psi}$ that reduces the advantage by at most
  $t \sum_{i \in
  \ell}\left(\frac{\varphi(\bar{m})}{f_i}p_i^{-(d-t)\cdot f_i} \right)$, where
  $f_i$ and $\bar{m}$ are as defined in Lemma ~\ref{lem:prob-relprime}.
\end{corollary}

In order for our reduction to ``first-is-errorless'' $\mlwe$ to be
meaningful when $d-t$ is much smaller than linear in the underlying
security parameter $\kappa$ (as it will be in the case we are
interested in for our reduction to $\mlwr$), we will require that all
of the prime ideals of $qR$ have norms superpolynomially large in the
security parameter (which makes the reduction in advantage
negligible). This is the case for many if not most practical choices
of $R$ and $q$. Similar conditions were required by Ducas and
Micciancio~\cite{DBLP:conf/crypto/DucasM14} for the ring used in their
signature scheme, although they were significantly further restricted
because they needed the prime ideals to be exponentially large in the
security parameter. As in their work, for a concrete example of
choices of $R$ and $q$ for which this reduction is meaningful, we can
take $R$ to have index $2^{k}$ and let $q=3^{\ell}$. In fact, for $R$
of index $2^{k}$, noting that $\Z_{2^k}^* \equiv \Z_2 \oplus
\Z_{2^{k-2}}$, we have asymptotically by the prime number theorem for
arithmetic progressions~\cite{soprounov2010short} that the density
of all primes (and their powers) that are admissible as prime factors of
the modulus $q$ for which the reduction is meaningful is $1-1/2^{k-1}$

\subsubsection{First-Is-Errorless to Extended-LWE}
\label{sec:first-errorl-extend}

We now show a reduction from ``first-t-are-errorless'' $\mlwe$ to
$\extmlwe$. The reduction primarily follows that of Brakerski et al.,
with slight differences to account for the multiple samples, as well
as a correction of some small mistakes in the proof therein. 
\begin{lemma}
Let $\calZ \subseteq K^d$ be of quality $(\xi > 0,t)$. Then for any $n, q
> 1, \epsilon \in (0,1/2)$, $\alpha, r \geq
(\ln(2w(1+1/\epsilon))/\pi)^{1/2}/q$, $t < d$, there is a (transformation)
reduction from the first-$t$-are-errorless variant of
$\mlwe_{R,d,w,q,\alpha}$ to
\\$\extmlwe_{R,d,w,q,(\alpha^2\xi^2+r^2)^{1/2},\calZ,t}$ that reduces
the advantage by at most $16\epsilon$. 
\end{lemma}

\begin{proof}
The reduction proceeds as follows. Given $\vecz_1, \ldots, \vecz_t \in
\calZ$, we compute the unimodular $\matU \in \Z^{d \times d}$ that can
be efficiently computed as in Definition~\ref{def:quality}, and let
$\matU' \in \Z^{d \times (d-t)}$ be the matrix formed by removing the
  $t$ leftmost columns of $\matU$. We receive $m$ samples $(\matA \in
  \R_q^{d \times w}, \vecb \in \R_q^{w})$ from our
  ``first-$t$-errorless'' $\mlwe$ oracle. We
  then sample 
$\vecf \gets D_{\alpha(\xi^2\matI-\matU\matU^t)^{1/2}}$. The above distribution is well-defined since
  $\xi^2\matI-\matU'\matU'^t$ is a positive semidefinite matrix
  according to our assumption on $\matU$. 

The reduction then sets $\bar{\vecb}=\matU\vecb+\vecf$, and samples 
$\vecc$ from the discrete Gaussian distribution
$D_{q^{-1}R^{w}-\bar{\vecb},r}$. It then outputs 
\[(\matA'=\matA\matU^t,\vecb'=\bar{\vecb}+\vecc,(\trace(\inner{\vecz_i,\vecf+\vecc}))_{i
  \in [t]}) \in R_q^{d \times w} \times R_q^{w} \times \Q_q^{t}\]

We now analyze correctness of the reduction. First, we have that
$\matU$ is invertible, so that $\matA'=\matA\matU^{t}$ is uniform
solely over the choice of $\matA$. For the remainder of the proof we
condition on any fixed value of $\matA'$ and focus on analyzing the
distribution of the second and third components of the output.

First, consider the case that the samples received by our reduction
came from the first-$t$-are-errorless variant of $\mlwe$. Then we have
that 
\[\bar{\vecb}=\matU\vecb+\vecf=\matA'^t\vecs+\matU\vece+\vecf \]

We have that $\matU\vece$ is distributed as a continuous Gaussian
$D_{\alpha\matU}$, so by additivity of Gaussians, we have that
$\matU\vece+\vecf$ is distributed as a spherical continuous Gaussian
$D_{\alpha\xi}$. Since $\matA'^{t}\vecs \in R_q^w$, we have
that $\vecc$ is in fact distributed according to
$D_{q^{-1}R^{w}-\bar{\vecb},r}$.  As a result, by
Lemma~\ref{lem:latt-facts}, 
$\matU\vece+\vecf+\vecc$ is within statistical distance $8\epsilon$ of
$D_{q^{-1}R^{w},(\alpha^2\xi^2+r^2)^{1/2}}$, which shows that $\vecb'$
is distributed correctly. For the third component, since
the first $t$ elements of $\vece$ are zero and the last $d-t$
columns of $\matU$ are orthogonal to each $\vecz_i$, we have that 
\[\inner{\vecz_i,\vecf+\vecc}=\inner{\vecz_i,\matU\vece+\vecf+\vecc},\]
so the third component gives the inner product of the noise with each
vector $\vecz_i$, as desired. 

Finally, we consider the case where the samples received by our
reduction are truly uniform. Then since $\vecb$ is truly uniform and
independent, we can view it as being distributed as
$\vecb=\tilde{\vecb}+\vece$, where $\tilde{\vecb} \in R_q^{w}$
is truly uniform and $\vece$'s first $t$ coordinates are $0$, and the
remaining coordinates are chosen independently from $D_{\alpha}$. Then
we have that $\vecb'=\matU\tilde{\vecb}+\matU\vece+\vecf+\vecc$, where
$\matU\vece+\vecf+\vecc$ is distributed exactly as in the case
above. Since $\matU$ is unimodular, we have that
$\matU(\tilde{\vecb}+\vece)$ is uniform and independent over
$R_q^{w}$. Finally, since $\matU\vece+\vecf+\vecc$ is distributed
exactly as above, we also have that the third component output is
distributed correctly, as needed.
\end{proof}
\fi
%%% Local Variables: 
%%% mode: latex
%%% TeX-master: "latticereduction"
%%% End: 
