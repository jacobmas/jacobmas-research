\subsection{Towards Relevance for NIST Candidate Algorithms}
\label{sec:towards-relev-nist}

\jnote{TODO: completely rewrite these were some probably unreadable
  thoughts when we were just going to write a note about the Indian
  guys work}
In terms of 2nd round submissions to NIST, Frodo appears to be the
only scheme to which this result is potentially relevant.

Round5, the
  other 2nd round NIST submission using unstructured lattices, is both
  based on the \emph{Learning with Rounding} problem and makes use of
  a sparse secret distribution, which appears to make Regev's
  reduction totally irrelevant.\footnote{this does not mean Round5 is
    insecure, only that it is much much farther from being covered by
    a reduction than Frodo is}

To attempt to determine what parameters Frodo needs to be scaled up to
in order to claim a non-vacuous reduction even in the random oracle
model, the reduction ought to be analyzed with respect to a proper
noise parameter $\alpha$ is the primary 

\jnote{TODO: describe why the LWE in the public key is the meaningful
  one for the reduction, e.g. if there exists an LWE adversary which
  breaks on the higher-noise samples in Enc, we can transform it into
  an adversary that breaks on the public key alone by creating the
  necessary additional samples from the public key}


In addition, in an actual run of the protocol, the ``bad cases'' of  

For the public key, 

%%% Local Variables: 
%%% mode: latex
%%% TeX-master: "regevreduction"
%%% End: 
