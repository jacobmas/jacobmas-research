\section{Main Idea Sketch}
\label{sec:mainidea}

\subsection{Heart of Idea}
\label{sec:heartofidea}




Letting
\[\matA=\begin{bmatrix}\bar{\matA}\\ \vecb^{t}\\\end{bmatrix}\] be the
public key where $\vecb^{t}=\vecs^{t}\matA+\vece^{t}$ and $\vecs$ be
the secret key, as usual in GSW-style encryption.

Let
$d=\log{m}=c\log{n}$ be the depth of the $\textsf{NC1}$ circuit to be
evaluated. We consider for simplicity an $\textsf{NC1}$ circuit where
every gate has fan-in two, fan-out 1. Due to the need to keep the
message at all points during the computation at magnitude at most
$1$ and for simplicity we limit the operations $\textsf{NAND}$ gates,
$x \textsf{ NAND }
y = (1-xy)$ \jnote{might be
  faster/lower depth
  to allow ANDs and NOTs alone}


\jnote{TODO: nice image of such a thing}

Begin with $m=\poly(n)$ initial ciphertexts. 
\begin{equation}\matC^{(0)}_{i}:=\matC_{i}=\matA\matS_{i}+\mu_{i}\matG\end{equation}


At depth $j$, define for $i \in [1,\ldots 2^{d-j}]$ to be 

\[\matC^{(j)}_{i}:=1-(\matC^{(j-1)}_{2i - 1} \otimes
  \matC^{(j-1)}_{2i}),\]
where $\otimes$ denotes the standard GSW homomorphic multiplication operation.


After evaluating the circuit, the resulting ciphertext $\matC^{*}$
encrypting $\mu^*$ (the function computed) is 


\begin{equation}\matC^{*}:=\matC^{(d)}_{i}:=\mu^{*}G+\sum_{i \in
    [m]}\tilde{\matC}^{(d)}_{i},\end{equation}

\paragraph{Term Equation.} 

Each term $\tilde{\matC}^{(d)}_{i}$ is
equal to 

\begin{equation*}\label{eq:term_eq}\tilde{\matC}^{(d)}_{i}:=\tilde{\mu}^{(d)}_{i}\matA\matS_{i}\matF^{(d)}_{i},\end{equation*}


\subsection{Zucking: Finding pre-images close to the Right-Side Terms}
\jnote{TODO: part about finding matrix $\matZ^{(d)}_{i} \in \lamperp(\matG)$
  such that $\length{\matZ^{(d)}-\matF^{(d)}_{i}}$ is short}

In this subsection, we describe how to find pre-images close to the
right-hand side noise of a single term of the form \jnote{multiplication noise? come up with
  term}. 

\begin{equation*}\label{eq:term_eq_2}
  \mu^{*}\matA\matS_{i}\matF^{*}\end{equation*}

given $\matF^{*}$ and a ciphertext encrypting 


\begin{equation*}\matC^{(0)}_{i}:=\matC_{i}=JUNK+mu^{*}\matA\matS_{i}+\tilde{\mu}\matG.\end{equation*}

For each column $\vecf^{*}_i$ of $\matF^{*}$, we find a $\vecz^{*}_{i}$ in $\lamperp(\matG)$  close to
$\vecf^{*}_{i}$. Note that except with negligible probability, $\matF^{*}$
will lie in a non-zero coset of $\lamperp(\matG)$. 
Although finding such vectors $\vecz^{*}_{i}$ in $\lamperp(\matG)$ is
hard in \emph{random lattices}, $\matG$ is designed specifically to
make tasks like this easy. 

In particular, the standard bit-decoding to find $\vecz^{*}_{i} \in
\bit^{m}$ such that 
\[\matG\vecz_{i}^{*}=\matG\vecf^{*}_i \bmod{q}\] should be
sufficient, see ~\cite{DBLP:conf/eurocrypt/MicciancioP12} for details
(\jnote{TODO: and add more})

\jnote{TODO: I had something written up for one of our non-published
  papers that should improve this a bit, should we add it?}



\jnote{TODO: ignore JUNK initially?}


\subsection{Path Through Circuit}
\label{sec:path}

\jnote{TODO: CAN'T HAVE R IN CIPHERTEXTS and message maybe change to $\matA\matS_{i}$}

Since the fan out is 1, there is a single path from each input to the
output (this fact appears to be crucial in our algorithm). In
particular, the terms $\matC^{(j)}_{i \shiftright j}$ for $j \in
[0,\ldots,d]$ make up the intermediate ciphertexts in the single path
to the final result.

Consider a term $\tilde{\matC}^{(d)}_{i}$

Let \[L_{i} = \{\matC^{(j)}_{(i \shiftright j)-1} : (i \shiftright
  j)\wedge 1=1), j \in [0,\ldots,d]\}\]
be the 'ordered set' of ``left siblings'' of ciphertexts in the path
for input $i$, and denote by $\matL_{i}^{\beta}$ the $\beta$th such
element in the path (proceeding normally in the input to output
direction). 

Similarly, let \[R_{i} = \{\matC^{(j)}_{(i \shiftright j)+1} : (i \shiftright
  j)\wedge 1=0), j \in [0,\ldots,d]\} \] be the ``ordered
set''\jnote{TODO: reword} of ``right siblings'' of
ciphertexts in the path for input $i$, and denote by $\matR_{i}^{\beta}$
the $\beta$th such element in the path. 

Then letting $\lambda_{i}=\abs{L_{i}}$, we have that, 

\begin{equation}\tilde{\mu}^{(d)}_{i}=\prod_{\beta \in
    [\lambda_{i},\lambda_{i}-1,\ldots,1]}
  \pkcdec(\matL_{i}^{\beta})\end{equation}

\jnote{explain why it's in backwards order}


Similarly, we have that
\begin{equation}\matF^{(d)}_{i}=\prod_{\beta \in
    [\abs{\matR_{i}}]} \matG^{-1}(\matR_{i}^{\beta})\end{equation}

Crucially, \emph{the evaluator can compute} the value of
$\matF^{(d)}_{i}$. 




\subsection{Unzucking/Blyat}


In parallel for each $i$, we first compute 

(in left-associative order should be fine since there
are only at most $d=\log{m}$ terms ) 
\[\matE_{i}:=\matL^{\lambda_{i}}\matG^{-1}(\matL^{\lambda_{i}-1}\matG^{-1}(\ldots
  \matL_{1}\matG^{-1}(\matC^{(0)}_{i}))\ldots))\]


Fuck that doesn't work does it blyat



%%% Local Variables:
%%% mode: latex
%%% TeX-master: "fheimprovements"
%%% End: