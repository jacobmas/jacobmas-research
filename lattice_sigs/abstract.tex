With the exception of ``Frodo,'' practical lattice-based cryptography is generally done over ideal lattices and module lattices. Nevertheless, other than for purely practical schemes, lattice-based cryptography papers are almost all written in terms of general lattices.  In this work, we aim to convince that rather than ``taking off the ring'', it is far more prudent to instead listen to Gollum and consider the ring setting to be ``precious.'' To do so, we give significant improvements to three lattice-based signature schemes by cleverly exploiting the ring setting. 

Our primary improvement is to the Boyen-Li signature scheme. Our improved scheme provides a verification key smaller by a linear factor, a significantly tighter reduction with only a constant loss, and signing and verification algorithms that could plausibly run in about 1 second. We change the scheme in a manner that allows us to replace the homomorphic pseudorandom function evaluation with an evaluation of a much more efficient weak pseudorandom function, and then take advantage of the fact that the ring setting allows for efficient homomorphic evaluation of this function. We then use the techniques developed in improving the Boyen-Li signature scheme to give improved versions of two other lattice-based signature schemes. In particular, these improved schemes are the first that have (nearly) optimally short public keys and short signatures with standard model security based on the hardness of polynomial approximations of worst-case lattice problems.  

As a matter of independent interest, we give an improved method of randomized inversion of the $\matG$ gadget matrix [MP12], which reduces the noise growth rate in homomorphic evaluations performed in a large number of lattice-based cryptographic schemes,  without incurring the high cost of sampling discrete Gaussians. 