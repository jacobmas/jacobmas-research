\section{Signatures With Short Public Keys}
\label{sec:sig-short-pub}
In this section we describe our two
lattice-based signature schemes with short public keys but non-tight
reductions. As it turns out, the use of the PHTDF paradigm allows us
to provide identical descriptions for both schemes in terms of the
function $g$ to be homomorphically evaluated, freeing us to spend most
of our exposition on the efficient evaluation of $g$. 

\subsection{Generic Signature Scheme}
\label{sec:sig-short-constructions}

We follow the presentation of
Alperin-Sheriff~\cite{DBLP:conf/pkc/Alperin-Sheriff15}. Scheme $b$ for
$b \in \{1,2\}$ is parameterized by a security parameter
$\lambda^{(b)}$ and a family of functions
$\{g^{(b)}: \calT^{\kappa} \times \calT^{\ell}\} \in \calG$ admissible
with parameter $s^{(b)}$.  Associated with each function $g^{(b)}$
will be a vector of tags (fixed over the instantiation of the scheme)
$\hat{T}^{(b)} = \hat{\vect}^{(b)} \in \calT^{\kappa}$ used
in the actual scheme and an additional set of tags
$\hat{T}^{(b)*} = \hat{\vect}^{(b)} \in \calT^{\kappa}$
used in the security reduction.

There will
also be a vector of tags $T^{(b)} \gets D_{\calT^{\ell}}$ to be
sampled. (see
Section~\ref{sec:tag-inst} for details).  Finally, we set the index space $\calX =
\{-1,0,1\}^{\bar{m}}$ and need a collision-resistant hash function
$h: \{0,1\}^* \to \{0,1\}^{\bar{m}}$, to ensure that any $x^* \in
\calX$ with a negative element is not a valid hash of any message.
To avoid further clunkiness, we omit the $(b)$ superscripts in the description
of the scheme below.

\begin{description}
\item[$\siggen(1^{\lambda})$:] Compute $pk \gets
  \phtdf.\tdfgen(1^{\lambda})$, let $v \gets \calV$, $(a_i, r_i)
  \gets \phtdf.\gentrap(pk,t_i)$ for $i \in [\kappa]$. Set $vk =
  (pk,\veca, v)$, $sk = (\vecr)$.
\item[$\sigsign(sk, x = h(\mu) \in \calX)$:] Sample $T \gets
  \calD_{\calT^{\ell}}$, compute $r^* \gets \evaltd(g,
  (\veca,\vecr),T)$, $a^* \gets \evalfunc(g,\veca,T)$.  Let $u \gets
  \phtdf.\tdfinv_{r^*,pk,a^*,x,s}(v)$. Output $\sigma = (u,T)$. 
\item[$\sigver(vk, x= h(\mu) \in X, \sigma = (u,T)$] Compute $a^* =
  \evalfunc(g,\veca,T)$, and verify that $f_{pk,a^*,x}(u)=v$,
  $\tdfprop(u) \leq s$. 
\end{description}

\subsection{Definition of $g$}
\label{sec:tag-inst}


\subsubsection{Sampled Tag Instantiation}
\label{sec:tag-inst-real}

In both schemes, each vector of sampled tags
$T^{(j)} \in \calT^{\ell}$ will consist of length $\alpha(i)$th prefixes
of a string $t = [t_0,\ldots, t_{\bar{}n-1}] \in \bit^{\bar{n}}$
(where $n=\varphi(m)$, $\bar{n}=n/2$ is the degree of the $m$th
cyclotomic polynomial), mapped to the ring element via
\[t_{i}(X) = \sum_{j < \alpha(i)} t_jX^{j} \in R_q.\]  In both schemes,
sampling a vector of tags according to $D_{T^{\ell}}$ will correspond
to sampling $t \gets \bit^{\bar{n}}$ uniformly at random and then
deriving the vector of tags as just described. The difference in the
form of the vectors of tags in the two schemes lies entirely in the
definition of $\alpha(i)$. In the Alperin-Sheriff scheme, which uses
``confined guessing''~\cite{bohl2013confined}, $\alpha(i) = c^i$ for
some constant $c > 1$, and we have $\ell = \ceil{\log_2{\bar{n}}}$
(where if $2^\ell > n$, the $\ell$th prefix is just the full string).
In the scheme of Miccancio and Peikert, which uses the ``prefix
technique''~\cite{DBLP:conf/eurocrypt/HohenbergerW09}, we just have
$\alpha(i)=i$, and we have $\ell=\bar{n}$.

Note again that in both cases, the full vector of tags can be computed from
the single bitstring $t \in \bit^{\bar{n}}$, so that the tag increases the
size of the signatures over $u$ alone additively by an amount linear in the security
parameter. 



In both schemes, we will have $\kappa = \ell+1$ and the function $g: \calT^{\kappa} \times \calT^{\ell}
\to \calT$ that we wish to evaluate over the
(hidden) fixed vector of tags $\hat{T} = (\hat{t}_0, \ldots \hat{t}_{\kappa-1=\ell}) \in \calT^{\kappa}$ and the (public) sampled
vector of tags $T = (t_1, \ldots, t_{\ell}) \in \calT^{\ell}$ can be defined as
\[g(\hat{T}, T) : = t_0 + \sum_{i \in [\ell]}\hat{t}_i\]



\subsection{Conditions on $g$ For Security}
\label{sec:cond-g-secu}

We recall the conditions that $g$ must satisfy in order for the
security proofs in~\cite{DBLP:conf/pkc/Alperin-Sheriff15} and
~\cite{DBLP:conf/eurocrypt/MicciancioP12} to go through, and briefly,
how those conditions are satisfied, referring readers to each of those
works for more details and for full security proofs.

As with the scheme in Section~\ref{sec:func-eval}, in both schemes we will set $\calT$
equal to $R_q$ for a cyclotomic ring $R = \calO_{m}$, where
$m=2^{k}$. However, here we need it to be the case, as in the earlier
work of Ducas and Micciancio  that the modulus $q=p^t$ for some prime $p$ such that the
$m$th cyclotomic polynomial (of degree $n=\varphi(m)$) factors modulo $p$ into
polynomials of degree as high as possible, namely $\bar{n}=n/2$, which will be
the case when $p=3$. For further details
on how cyclotomic polynomials factor over a finite field,
consult, e.g.~\cite[Theorem 2.47]{FiniteFieldsIntro}.


In both schemes, we need $g: \calT^{\kappa} \times \calT^{\ell} \to
\calT$ to fulfill certain properties when it is homomorphically evaluated $Q$ times (corresponding to
$Q$ signature queries) on the fixed set of tags $\hat{T}$ and on each
sampled set of tags $\calT \gets D_{T^{\ell}}$.

For both actual schemes, $g(\hat{T},T_i)$ must always be an invertible
element of $\calT$ for each of the $Q$ queries.
In addition, for any given $\hat{T}$, it should be possible to find in
polynomial time some $\bar{T}$ such that $g(\hat{T},\bar{T})=0$. 

The conditions for the security reductions in both schemes are quite
similar as well. We need that with non-negligible
probabilities, given the $Q$ vectors of tags $T_i$ sampled
to create signatures, we can find in
  polynomial time a $\hat{T}$ such that
\begin{enumerate}
\item \begin{description}
\item[Scheme 1:] except for at most one $i$, we have that
  $g(\hat{T}, T_i)$ is invertible. 
\item[Scheme 2:] for \emph{all} $i$, we have that $g(\hat{T}, T_i)$ is
  invertible.
\end{description}
\item The tag $T^*$ chosen by the adversary for his forgery is such
  that $g(\hat{T}, T^*)=0$.
\end{enumerate}

We also briefly recall (see~\cite{DBLP:conf/eurocrypt/MicciancioP12}
and~\cite{DBLP:conf/pkc/Alperin-Sheriff15} for details) the
probabilities of fulfilling the above security conditions. In the
Micciancio-Peikert scheme, using the ``prefix
technique''~\cite{DBLP:conf/eurocrypt/HohenbergerW09} therein,
condition 1 is fulfilled with probability $\epsilon_1=1$ and condition
2 is fulfilled with probability $\epsilon_2=1/((\ell-1)*(Q+1))$. In
the Alperin-Sheriff scheme, using the ``confined guessing'' technique~\cite{bohl2013confined}
against an adversary with success probability $\delta$ and making $Q$
queries, condition $1$ will be fulfilled with probability at least
$\epsilon_1=1-\delta/2$, while condition $2$ is fulfilled with
probability at least $\epsilon_2=(\delta/(4Q^2))^c$, where $c>1$ is the constant defined in Section~\ref{sec:tag-inst}.




\subsection{Ring-Based Instantiations of $g$}
\label{sec:ring-based-inst-1}

In each of these schemes, the function $g$ that is being evaluated
homomorphically over tags in the verification key is linear over $\ell+1$
tags $\hat{T}=(\hat{t}_0,\hat{t}_1, \ldots, \hat{t}_{\ell})$. Specifically, letting $T= (t_1, \ldots t_{\ell})
\gets D_{\calT^{\ell}}$ be the tags sampled for a given signature, the
function $g$ was 
$g(\hat{T},T) := \hat{t}_0 + \sum_{i \in [\ell]}\hat{t}_i\cdot t_i$.

In each of the actual schemes, they set $\hat{t}_0 = 1, \hat{t}_i = 0$
for $i \in [\ell]$, which ensured that all messages could be
successfully signed. In the security reduction,  letting
$t^*_{\leq i^*}$ be the prefix of length $i^*$ of the tag which
they hoped the adversary would use in their forged signature, they set
$\hat{t}_0 = -t_{\leq i^*}^*$, $\hat{t}_i = 1$ for $1 \leq 1 \leq i^*$
and $\hat{t}_i=0$ for $i^* < i \leq \ell$.  

Using our Lagrange interpolation technique, we can limit the
verification key to two trapdoor functions $a_0$ and $a_1$, such that
$\hat{t}_0=1$, $\hat{t}_1=0$ in the actual scheme, and such that
$\hat{t}_0 = -t_{\leq i^*}^* \in R_q$ (as in the original schemes) and
such that $\hat{t}_1 = \zeta_{m}^{i^*} \in R_q$.  We have that
$\hat{t}_1$ will either be a power of
$\zeta^{m}$ (in the security reduction) or $0$ (in the actual
scheme). By Lemma~\ref{lem:lagrangefunc}, the $\hat{t}_i$ that
we want to homomorphically compute to reduce the rest of the
computation to a mere homomorphic summation can be computed entirely
from the tag encoded in $a_1$. In particular, for the associated
trapdoors $r^*_i$, we will have that
$\tdfprop(r_i^*) \leq \sqrt{\ell} \sqrt{n}\tdfprop(r)$. Thus, summing
$(\ell+1)$ elements, the trapdoor $\tilde{r}$ of the final output
trapdoor function from evaluating $g(\hat{T},T)$ homomorphically will
be such that
$\tdfprop(\tilde{r})\leq \sqrt{(\ell+1)\ell} \sqrt{n}
\tdfprop(r)$. This leads to the following lemma.

\begin{lemma}\label{lem:effvalcosting}
If each $r_i$ in the signing key is such that $\tdfprop(r_i) \leq
\tilde{O}(\sqrt{n})$ and $\tilde{n}=O(n),\tilde{q}=m=2n$, then $g$ is admissible with parameter 
$s=\sqrt{(\ell+1)\ell}\sqrt{n}\tilde{O}(\sqrt{n})$. 

Furthermore, by  Theorem~\ref{thm:phtdf}, we then have that the
improved version of the Micciancio-Peikert signature scheme is secure as long as
$\rsis$ is secure over the
$m$th cyclotomic ring with parameter
  $\beta=\tilde{O}(n^{9/2})$ and the improved version of the
  Alperin-Sheriff signature scheme is secure as long as $\rsis$ is
  secure over the $m$th cyclotomic ring with parameter $\beta = \tilde{O}(n^{5/2})$.
\end{lemma}


%%% Local Variables:
%%% mode: latex
%%% TeX-master: "newlattsigs"
%%% End:
