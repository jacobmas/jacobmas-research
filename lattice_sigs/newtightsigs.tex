\section{Improved Signature Scheme With Tight Security}
\label{sec:func-eval}

We give our improved signature scheme, and prove
security with a tight reduction, writing
the scheme generically in terms of the PHTDF cryptographic primitive,
as well as in terms of a generic weak PRF and chameleon hash
function. We
focus on the homomorphic evaluation details in Section~\ref{sec:weak-prf-evaluation}.

\subsection{Parameters}
\label{sec:sig-params}
All parameters in the scheme are in terms of an underlying security
parameter $\lambda$, with
$\alpha,w=\theta(\lambda)$. $\calT=R_q$. 
We need $\weakprf: \calT^{\kappa-2} \times \calT^{w-1} \to \bit$ be an
$(\epsilon_{\weakprf},t_{\weakprf})$-secure weak pseudorandom
function, where $\calT^{\kappa-2}$ is the keyspace and $\calT^{w-1}$ is the
input space. The value of the $\kappa\geq 2$ depends on the weak PRF used.
We need $\weakprf(\veczero,\cdot)=0$ for all
$\vecy \in \calT^{w-1}$, which is the case for $\lwr$.
 $\ch = (\chgen,\chhash,\chinv)$ is a secure chameleon
hash function family such that the input space $\calX$ is efficiently
sampleable and grows in size as a
(polynomial) function inthe underlying security parameter $\lambda$
and such that the output space
$\calY=\calT^{w-1}$. This can be achieved essentially without loss of
generality, as long as there is an efficiently
computable embedding of the output space $\calY$ into
$\calT^{w-1}$.
We define $g:\calT^{\kappa} \times \calT^{w} \to \calT$ as
$g((\vecs,z^{(0)},z^{(1)}),(\vecy,b))=(1-z^{(b)})-\weakprf(\vecs,\vecy)$,
where $\vecy \in \calT^{w-1}$ will be the output of a
chameleon hash function, and $\vecs \in \calT^{\kappa-2}$ will be the
secret key for the weak PRF. 



\subsection{Construction}
\label{sec:construction}
We need $\phtdf=(\phtdf.\tdfgen, \phtdf.\gentrap,$\\ 
$\phtdf.f,\phtdf.\tdfinv,\phtdf.\evaltd,\phtdf.\evalfunc)$ to be an $(\epsilon_{\phtdf},t_{\phtdf},g,\calS)$
CRP-secure PHTDF family, where $\calS=\calT^{\kappa-2}$, and require the input space $\calX$ for the chameleon
hash function to satisfy shortness properties (and be such that $\abs{\calX}=2^{\Omega(\lambda)}$), which is indeed the
case for the Ducas-Micciancio chameleon hash function family construction.

\paragraph{Encoding the PRF secret key.}\label{sec:prf-sec-encoding}
Often, including in the instantiation we detail below in Section~\ref{sec:weak-prf-evaluation}, the
way of encoding the weak PRF secret key $\vecs$ into the verification key
will be something other than its ``natural representation,'' and we
abstract this out with the additional utility function $\tdfencode:\calT^{\kappa} \to \calT^{\ell}$.



\begin{description}
\item[$\siggen(1^{\lambda})$] On input security parameter $\lambda$,
\begin{enumerate}
\item Choose $v \gets \calV$, $\vecs\gets
  \calT^{\kappa-2}$. Set $z^{(0)}=0,z^{(1)}=1$. Compute $\tilde{\vecz}
  = \tdfencode(\vecs, z^{(0)}, z^{(1)}) \in \calT^{\ell}$, $(ek,td) \gets \ch.\chgen(1^{\lambda})$, $pk \gets \phtdf.\tdfgen(1^{\lambda})$, $\{(a_i, r_i) \gets \phtdf.\gentrap(pk,\tilde{\vecz}_i)\}_{i \in [\ell]}$.
\item Output $vk=(pk,\veca=\{a_i\}_{i \in [\ell]},ek,v)$, $sk=(\vecr=\{r_i\}_{i \in [\ell]},td,\vecz)$
\end{enumerate}
\item[$\sigsign(vk,sk,\mu)$] On input message $\mu \in \bit^{\alpha}$, secret key $sk$,
  verification key $vk$:
\begin{enumerate}
\item Sample  $x
  \gets \calX$, compute the output $\vecy=\ch.\chhash(x,\mu)$, and
  then evaluate the weak PRF at $\vecy$ to get $b=\weakprf(\vecs,\vecy)$.
\item Compute $r^* \gets \evaltd(g, (\veca,\vecr),(\vecy,b))$, $a^* \gets \evalfunc(g,\veca,(\vecy,b)).$
\item 
Compute $u \gets \phtdf.\tdfinv_{r^*,pk,a^*,x,s}(v)$. Output
  $\sigma=(u,x,b)$.
\end{enumerate}
\item[$\sigver(vk,\mu,\sigma)$] On input message $\mu \in \bit^{\alpha}$,
  verification key $vk$, signature $\sigma=(u,x,b)$:
\begin{enumerate}
\item Compute $\vecy \gets \ch.\chhash(x,\mu)$, $a^* \gets \evalfunc(g,\veca,(\vecy,b))$
\item 
Verify that $\phtdf.\tdfprop(x)$,
$\phtdf.\tdfprop(u) \leq s$,$\phtdf.f_{pk,a^*,x}(u)=v$.
\end{enumerate}
\end{description}

% \paragraph{Correctness.} Since $b=\weakprf(s,\vecy)$, we have that
% $g(s,(\vecy,b))=1-b-b=1-2b=\pm 1$ whenever $b \in \bit$. Since this will
% always be an invertible element of the ring $\calT=R_q$, we have that
% $\phtdf.\tdfinv$ will invert successfully, and so verification will
% succeed with all but negligible probability.


\subsection{Security}
\label{sec:security}

\paragraph{Security.} We now prove that the signature scheme in Section~\ref{sec:construction} is
  secure.
\begin{theorem} Let $\Adv$ be a PPT adversary which breaks eu-acma
  security of the scheme in Section~\ref{sec:construction} in time $t$
  with advantage $\epsilon$. Then there exists an $\Adv'$ which runs
  in time $t+\poly(\lambda)$ and breaks the security of one
  of the weak PRF, the chameleon hash function and the PHTDF with
  advantage at least $\epsilon/6$.
\end{theorem} 
\begin{proof}
Before beginning the game with the adversary, we first classify a successfully forging adversary into one of three
mutually exclusive categories, which we denote as $\calC = \{\text{W-PRF}, \text{CH},
\text{CRP}\}$. 

Then, before instantiating the security game with the adversary, we
guess uniformly at random which of these three categories the
adversary falls into. We then run the security game corresponding to
our guessed category against the adversary. 

Let $\epsilon_{c}$ denote the advantage achieved by an
adversary who successfully forges and falls into category $\calC$. Then we
have that $\sum_{c \in \calC}\epsilon_{c} = \epsilon$. 
Furthermore, let
$\delta_{c}$ denote the conditional advantage that we are able to
successfully achieve in
attacking the underlying hard problem given that we run the attack for
category $c$ against a successfully forging adversary who falls into
category $c$. Then the overall advantage we achieve will be 

\[\frac{1}{3}\sum_{c \in \calC}\delta_{c}\epsilon_{c} \geq
  \frac{\epsilon}{3} \cdot \min_{c \in \calC} \delta_{c}.\]
Showing that $\delta_c \geq \tfrac{1}{2}$ in
all cases then proves the theorem.

 \paragraph{Weak PRF.} We consider the first of the three
 mutually exclusive categories that we guess the adversary will fall
 into. In this category, we have that the successfully forging adversary outputs $\mu$,
  $\sigma=(u,x,b)$ such that $\weakprf(\vecz,(\chhash(x,\mu))=b$. 

If this is our guessed category, we change our behavior in the
  signature scheme's security game as follows:
\begin{enumerate}
\item
We set the secret key $\vecz=(\vecs,\vecz_{k-1},\vecz_{k})$ in our scheme
  to a dummy secret key of $(\veczero,0,0)$ in $\siggen$, which will
  always allow us to sign regardless of the value of the
  $b=\weakprf(\vecs,\vecy)$. 
Since the $a_i$ output by $\phtdf.\gentrap$ are statistically uniform and independent, and everything
else in the public key is completely independent of $\vecz$, this is
  statistically indistinguishable. 
\item For a message query, we first query
  $\weakprf$ challenger for a uniform
  $\vecy \in \calT^{w-1}$, $b=\weakprf(\vecs,\vecy)$ for some unknown
  $\vecs$, and compute $x \gets \chinv(\mu,\vecy)$, which is
  statistically indistinguishable. Since we will
  have that $g(\vecz,(\chhash(x,\mu),b))=1-z_{\kappa-b}-0=1$, we can invert
  successfully and sign as usual.
\end{enumerate}
We send the $\mu^*, b^*$ from the forgery to the
  challenger, and succeed in guessing the output of the $\weakprf$ at
  $\mu^*$ with (conditional) probability $1$ and advantage
  $\frac{1}{2}$. 

\paragraph{Chameleon Hash Collision.} Next, we consider the second of
the mutually exclusive categories the adversary may fall into. In this
category, $\Adv$ fails to predict the
weak PRF but successfully outputs $\mu^*,\sigma=(u^*,x^*,b^*)$ such that
$\chhash(x^*,\mu^*)=\chhash(x',\mu')$ for some $x',\mu'$ it already
queried on.  If this is our guessed category of adversary, we change our behavior in the security game as follows:
\begin{enumerate}
\item In $\siggen$, we receive the evaluation key $ek$ for the
  chameleon hash function from a challenger. The public key itself remains
  exactly the same.
\item In Step 1 of $\sigsign$, we sample $x \gets
\calX$, and then compute $\vecy = \chhash(x,\mu)$, 
statistically indistinguishable by the chameleon hash's uniformity
property.
\end{enumerate}
When $\Adv$ outputs its  forgery such that 
$\chhash(x^*,\mu^*)=\chhash(x',\mu')$, we send
$x',x^*,\mu',\mu^*$, breaking the collision-resistance
of the chameleon hash function with conditional advantage $1$. 

\paragraph{PHTDF Collision When Punctured.} Finally, we consider the
third category of adversary. In this category, $\Adv$ has
forged successfully, but has neither successfully predicting the weak
PRF output nor output  a chameleon hash collision. If this is our
guessed category, we change our behavior in the security game as follows:
\begin{enumerate}
\item During $\siggen$, we choose $\mu'$ uniformly at
  random. We then sample
  $u' \gets D_{\calU}$, $x' \gets \calX$, let
  $\vecy'=\ch.\chhash(x',\mu')$, $b=\weakprf(\vecs,\vecy')$, and compute
  $a' \gets \evalfunc(g,\veca,(\vecy',1-b))$. 
Finally, we set
  $v \gets f_{pk,a',x'}(u')$, and hold onto $u'$ to use in forming our
  collision. By the statisticial distributional
  equivalence of inversion property of the PHTDF, this is
  statistically indistinguishable from the real game.
\end{enumerate}
If the adversary outputs a successful forgery
$(\mu^*,\sigma^*=(x^*,u^*,b^*))$, where
$\weakprf(\vecs,(\chhash(x^*,\mu^*))=1-b$, then with all but 
probability $1/\abs{\calX}=\negl(\lambda)$, $x^* \neq x'$. If they are in fact not equal, we can
output $u',x',\vecy',u^*,x^*,\vecy^*=\chhash(\mu^*,x^*)$ as our collision,
breaking the CRP security of the underlying PHTDF except with
probability negligible in $\lambda$, giving conditional advantage
$1-\negl{\lambda}$. 

\end{proof}
%%% Local Variables:
%%% mode: latex
%%% TeX-master: "newlattsigs"
%%% End:
