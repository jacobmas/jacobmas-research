\section{Efficient Evaluation of $g$ over Cyclotomic Rings}
\label{sec:weak-prf-evaluation}

In this section we describe how working over $R_q$ where $R$ the $m=2^{k}$th cyclotomic ring and $q=3^{\ell}$  allows us to
efficiently evaluate the
function $g(\vecz,(\vecy,b))=(1-z_{2-b})-\weakprf(\vecz,\vecy)$
 for
the specific case that $\weakprf$ is the $\lwr_{\tilde{n},\tilde{q},2}$
function for $\tilde{n}=O(\lambda)$ and $\tilde{q}=m$. 




\subsection{Encoding $\vecz$ and Computing the Linear Product}
\label{sec:encoding-z-into}

We encode each element $z_i$  of $\vecz \in
\Z_{\tilde{q}}^{\tilde{n}}$ as the tag for the trapdoor function $a_i$ of the
verification key
as $t_i=\zeta_{m}^{z_i}$. Then, to evaluate $\inner{\vecz,\vecy}$ over a (publicly
known) $\vecy \in \Z_{\tilde{q}}^{\tilde{n}}$, we simply use the
asymmetric multiplication technique from
Section~\ref{ref:concrete_inst} to homomorphically compute the product 
$t^*=\prod_{i \in [\tilde{n}]} (\prod_{j=1}^{y_i} t_i)$. 
As this product contains at most $\tilde{n}\tilde{q}$ terms, by
Theorem~\ref{thm:asym-growth} we have that the new trapdoor $r^*$
resulting from the homomorphic evaluation of the product will be such
that $\tdfprop(r^*) \leq 2\sqrt{\tilde{n}\tilde{q}}\sqrt{n}\max(\tdfprop(r_i))$.


\subsubsection{Further Noise Growth Reduction.}
\label{sec:further-efficiency}

To further reduce the noise growth in the context of $\lwr$ evaluation, instead of
homomorphically computing just $t^*$ in this step, we can instead efficiently
compute, separately, $c_{k}(t^*)^{k}$ for $k \in
\{0,\ldots, m-1\}$ for an appropriate constant $c_{k}$. To do so, we
still use the left-associative asymmetric
multiplication technique, and compute the product as
$(t^*)^k=\prod_{i \in [\tilde{n}]} (\prod_{j=1}^{(y_i * k)
  \bmod{\tilde{q}}} t_i) \cdot c_{k}$. Each new trapdoor $(r^*)^{(k)}$ resulting from this
homomorphic computation of $(t^*)^k$ will be such that $\tdfprop(r^*)
\leq 2\sqrt{(\tilde{n}\tilde{q}+1)}\sqrt{n}\max(\tdfprop(r_i))$. 



\subsection{Lagrange Interpolation over
  $R_q$}\label{sec:lagr-interp-over}


\subsubsection{Proofs of Feasibility and Homomorphic Cost.}
\label{sec:proofs-feasibility}

To evaluate the rounding part of the $\lwr_{\tilde{n},\tilde{q},2}$ function homomorphically,
we will need to be able to evaluate a function over $R_q$ that sends $\zeta^i$ to
$0$ when $i \in [-\tilde{q}/4,\tilde{q}/4)$, and sends $\zeta^i$ to $1$ otherwise.    

If $R_q$ were a finite field, we could immediately do so with Lagrange
interpolation. However, it is known that the $2^k$th cyclotomic polynomials are
reducible modulo every prime. 
While we could simply work over an embedded subfield $P^{\ell}$ of
sufficiently high degree $\ell$, this would either limit our choice of
rings or require a significant loss in parameters. Instead, we show
that Lagrangian interpolation will work over the set of input points
$\{\zeta_{m}^{i}\}_{i \in [m]}$ in $R_{q}$ as long as $q$
and $m$ are coprime. 

\begin{lemma}\label{lem:prod_stuff} $\prod_{i=1}^{m-1}(1-\zeta^i) = m$.
\end{lemma}
\begin{proof}
We have that $X^{m}-1=(X-1)(X-\zeta)\ldots (X-\zeta^{m-1})$, so we
have that $\sum_{i=1}^{m-1}(X-\zeta^{i}) = (X^{m}-1)/(X-1)=\sum_{i=0}^{m-1}X^{i}$. Evaluating this at $X=1$
gives the required result.
\end{proof}

By multiplying each of the $m-1$ terms in the above product by
$\zeta^{j}$, we get 
\begin{corollary}\label{cor:invertibledenom}
For any integer $j$,
w, $\prod_{i \in [m], i \neq j}(\zeta^j-\zeta^i) = \zeta^{-j}m$.
\end{corollary}

As a result, as long as $m$ is invertible modulo $q$, the product of
the differences will be invertible. For use in
Section~\ref{sec:sig-short-pub}, we note that $\zeta^{j}$ itself is
always invertible, which allows the use of $0$ as an input too. 
As a result,  for any function $f: \{\{0\} \cup \{\zeta_m^{i}\} \in R_q: \in \Z\} \to
\{0,1\}$, we can use Lagrange interpolation to compute any function taking
roots of unity to $\{0,1\}$, including the one required to evaluate the
rounding part of the $\lwr$ function.
%  and can then make the coefficients of
% the polynomial part of the public key, or even have the signing and
% verification algorithms compute the interpolation polynomial on the fly if we
% wish to have a smaller public key. 

We now determine the cost of evaluating this polynomial homomorphically in terms of
growth of the underlying trapdoors. Evaluating the polynomial 
$f(X) = c_{m-1}X^{m-1}+\ldots +c_{1}X+c_0$ in general would involve
first homomorphically evaluating each monomial via asymmetric
multiplication of $m$ terms, and then homomorphically summing the
results. By Theorem~\ref{thm:asym-growth}, we have that the resulting trapdoor
$r^*$ will be such that $\tdfprop(r^*) \leq m\sqrt{n}\tdfprop(r)$, where $r$ is the
trapdoor encoding the input value of $X$ to the polynomial. 

Combining this all, we have the following lemma.

\begin{lemma}\label{lem:lagrangefunc}Let $f: \{\zeta_m^{i} \in
  R_q: i \in \Z\} \to \{0,1\} \in R_q$. 
Then $f$ can be evaluated as a degree $m-1$ polynomial over $R_q$ and
its coefficients
can be efficiently computed. Furthermore, when
homomorphically evaluated over a trapdoor function with trapdoor
$r$, the trapdoor $r^*$ for the result will be such that $\tdfprop(r^*)
\leq m\sqrt{n}\tdfprop(r)$. 
\end{lemma}

 




\subsubsection{Homomorphic Evaluation of the Rounding Function.}
\label{sec:lwr-function}

Using our idea from Section~\ref{sec:encoding-z-into}, we have a more
efficient way of evaluating the rounding function. Given functions and
trapdoors $a_k, r_k$ encoding $c_k(t^*)^k$ from above for $k \in
\{0,\ldots,m-1\}$, evaluation will only require the homomorphic summation of the
results, which gives that

\begin{lemma}\label{lem:effevalcost}
If each $r_i$ in the signing key is such that $\tdfprop(r_i) \leq
\tilde{O}(\sqrt{n})$ and $\tilde{n}=O(n),\tilde{q}=m=2n$, then $g$ is admissible with parameter 
$s=4\sqrt{\tilde{n}\tilde{q}}\sqrt{n}\tilde{O}(\sqrt{n}) =
\tilde{O}(n^{2})$.

Furthermore, by  Theorem~\ref{thm:phtdf}, we then have that the signature
scheme in Section~\ref{sec:func-eval} is secure as long as
$\lwr_{n,\tilde{q},2}$ is secure for
$\tilde{n}=O(\lambda),\tilde{q}=m$, and  $\rsis$ is secure over the
$m$th cyclotomic ring with parameter
  $\beta=\tilde{O}(n^{9/2})$.
\end{lemma}


%%% Local Variables:
%%% mode: latex
%%% TeX-master: "newlattsigs"
%%% End:
