\section{Introduction}
\label{sec:intro}

The failure boosting technique can in some sense be
considered the only possible avenue of attack against an
IND-CPA secure scheme with decryption failure that has been
transformed into an IND-CCA secure scheme using the Fujisaki-Okamoto transformation~\cite{DBLP:conf/tcc/HofheinzHK17,DBLP:conf/crypto/JiangZCWM18}. This
technique was first used in an ad-hoc manner in CCA attacks
against schemes with decryption failures in \jnote{TODO:cite, e.g. PQC
forum maybe} and
formally described by D'Anvers et
al.~\cite{DBLP:conf/pkc/DAnversGJNVV19}. 

In many encryption schemes,
including many submitted to the NIST PQC standardization process,
causing a sufficiently large number of
decryption failures allows an adversary to recover the secret
key\jnote{TODO: cite Fluhrer etc}. The idea of failure boosting comes from the fact
that in a practical scenario\jnote{TODO: cite NIST}, an adversary is likely to be able to
do significantly more offline computing than online
computing actually mounting an attack against a specific scheme
instantiation. The idea behind failure boosting is to do a significant
amount of precomputation (via many evaluations of the secure cryptographic
hash function used for the Fujisaki-Okamoto transform) to derive a
large set of ``weak messages,'' that is, messages that will be more
likely (over the randomness used in the generation of the public and
secret key and [if any] in the decryption process) to result in a
decryption failure. 

Concretely, in terms of schemes submitted to the NIST PQC
standardization process, a scheme needs to remains secure against up
to $2^{64}$ CCA queries in any of the categories, while it needs to
 be secure against an adversary able to do up to $2^{128},2^{192},2^{256}$
total computation to be meet the requirements of NIST security categories 1,3,5
respectively. \jnote{etc etc}


%%% Local Variables: 
%%% mode: latex
%%% TeX-master: "round5vuln"
%%% End: 