\newcommand{\bbF}{\mathbb{F}}
\section{Introduction}

\subsection{Basics of Scheme}
\label{sec:basics-scheme}



In the simplified version of UOV where the polynomials are all
homogenous, we have that each ``secret'' polynomial can be represented by 

\[\matF^{(i)} = \begin{bmatrix}\matzero&\matA^{(i)}\\\matB^{(i)} &
  \matC^{(i)}\\\end{bmatrix} \in \mathbb{F}^{(n+v) \times (n+v)},\]

where $\matA^{(i)} \in \bbF^{n \times v}, \matB^{(i)} \in \bbF^{v
  \times n}, \matC^{(i)} \in \bbF^{v \times v}$, and we have 

\[\matG^{(i)}=\matS^{t}\matF^{(i)}\matS,\]

where 
\[\matS
= \begin{bmatrix}\matS_{1,1}&\matS_{1,2}\\\matS_{2,1}&\matS_{2,2}\\\end{bmatrix} \in
  \bbF^{(n+v) \times (n+v)}\]
 is invertible. 




Recall that to sign a message $\vecm \in \F^{n}$, the signer (who
knows the ``hidden'' matrices $\matF^{(i)}$) first chooses
$\vecw^{(2)} \gets \mathbb{F}^{v}$ uniformly at random
and then computes 

\[\vecv_i^t := (\vecw^{(2)})^t(\matB^{(i)}+(\matA^{(i)})^t) \in \F^{1
  \times n}, \quad m'_i := m_i -
(\vecw^{(2)})^t\matC^{(i)}\vecw^{(2)} \in \F\]

Letting $\matV = \begin{bmatrix}\vecv_1^{t}\\ \ldots \\ \vecv_n^t
  \\\end{bmatrix}$, we can then find a full (pre)-signature $\vecw \in
\F^{v+n}$ by solving the equation

\begin{equation}\label{eq:solvesystem}
\matV\vecw^{(1)}=\vecm'\end{equation}
\emph{if a solution exists}. If a solution
does not exist, the signer must repeat the procedure by sampling $\vecw^{(2)}$
again and checking again. 

It is important to note that there is not a unique secret $\matS \in
\F^{(n+v)\times (n+v)}$ that induces each hidden $\matF^{(i)}=\matS^{-t}\matG^{(i)}\matS^{-1}$
into having the proper form with the top-left corner equal to
$\matzero$. Indeed, it is easy to see that we can let
$\matS=\matT\matS'$, where
$\matT=\begin{bmatrix}\matW&\matY\\\matzero&\matZ\\\end{bmatrix} \in
\F^{(n+v) \times (n+v)}$,  for any invertible $\matW \in \F^{n \times
  n}, \matZ \in \F^{v \times v}$ and any $\matY \in \F^{n \times v}$,
and we will have that $\matF'^{(i)} =\matT^{t}\matF^{(i)}\matT$ will
have the top-left corner equal to $\matzero \in \F^{n \times n}$. 

In particular, whenever there is a solution to $\matS_{2,2}\matX=\matS_{2,1}$ (which
should be the case with probability roughly $1-1/q$ if $\F=\text{GF}(q)$),we may set
$\matS'=\begin{bmatrix}\matI&\matzero\\\matX&\matI\\\end{bmatrix}$, in
which case
$\matS'^{-1}=\begin{bmatrix}\matI&\matzero\\-\matX&\matI\\\end{bmatrix}$. We
will assume that the secret key $\matS$ is of this special form for
the remainer of this work. 

Finally, to compute the actual signature $\vecu$ from $\vecw$, the signer then
computes $\vecu=\matS^{-1}\vecw$. 

\subsection{An Attacker?}
\label{sec:an-attacker}

The key hail mary hope here is that knowledge of the distribution of the signatures
received will itself leak $\matS'$, because we know how each signature
was devised and we know that Equation ~\ref{eq:solvesystem} has a
solution. 
Let $\vecu^t = [(\vecu_1^t \mid (\vecu_2^t)]$ be a signature. 


Let 

\[\veca_i^t =
\vecu_1^t(\matG_{1,1}^{(i)}+(\matG_{1,1}^{(i)})^t)+\vecu_2^t(\matG_{2,1}^{(i)}+(\matG_{1,2}^{(i)})^t)
\in \F^{1 \times n}\]
and
\[\vecb_i^t=\vecu_1^t(\matG_{1,2}^{(i)}+(\matG_{2,1}^{(i)})^t)+\vecu_2^t(\matG_{2,2}^{(i)}+(\matG_{2,2}^{(i)})^t)
\in \F^{1 \times v}\] 
and 
\[c_i=\vecu_1^t(\matX^t\matG_{2,2}^{(i)}\matX)\vecu_1+\vecu_1^t\matX^t\matG_{2,2}^{(i)}\vecu_2+\vecu_2^t\matG_{2,2}^{(i)}\matX\vecu_1+\vecu_2^t\matG_{2,2}^{(i)}\vecu_2\]
Then (recalling the definition of $\vecv_i$ from above)
\[\vecv_i=\veca_i^t+\vecb_i^t\matX\]
\[m_i' = m_i - c_i\]

So we end up with

\begin{equation}\label{eq:secondnew}
\matV\vecu_1 = \vecm'
\end{equation}

Letting $\vect_1=\matX\vecu_1$ be the unknown variable, we then have 

Since $\vect_1$ is linear 
we can end up with 

\[a_i' =
\left((\vecu_1^t(\matG_{1,1}^{(i)}+(\matG_{1,1}^{(i)})^t)+\vecu_2^t(\matG_{2,1}^{(i)}+(\matG_{1,2}^{(i)})^t))\right)\vecu_1
\in \F^{1 \times n}\]
and
\[b'_i=\vecu_1^t(\matG_{1,2}^{(i)}+(\matG_{2,1}^{(i)})^t)\vect_1
\in \F^{1 \times v}\] 
and 
\[c'_i=\vect_1^t\matG_{2,2}^{(i)}\vect_1+\vecu_2^t\matG_{2,2}^{(i)}\vecu_2\]
Then we still have that 

\[\veca'+\vecb' = \vecm - \vecc',\]

so that finally (NOTE: $\vecu$s should have $j$ superscripts too) , letting 
\[(\vecr^{(j)}_i)^t=\vecu_1^t(\matG_{1,2}^{(i)}+(\matG_{2,1}^{(i)})^t)\]

\[s^{(j)}_i=m^{(j)}_i-\vecu_2^{t}\matG_{2,2}^{(i)}\vecu_2^{t}-\left((\vecu_1^t(\matG_{1,1}^{(i)}+(\matG_{1,1}^{(i)})^t)+\vecu_2^t(\matG_{2,1}^{(i)}+(\matG_{1,2}^{(i)})^t))\right)\vecu_1
\]


we have 

\[(\vect^{(j)}_1)^{t}\matG_{2,2}^{(i)}\vect^{(j)}_1+\inner{\vecr^{(j)}_i,\vect_1^{(j)}}=s^{(j)}_i\]

which substituting for $\vecu_1$ is a quadratic equation in the
$\matX$ variables. 

In theory, enough would allow linearization ...

But wait, this could be found ourselves with no oracle queries by
simply choosing a lot of $\vecu$s uniformly at random and re-writing. 

I bet it doesn't work because they're not in the right space or
something? 



%%% Local Variables:
%%% TeX-master: "mqstuff.tex"
%%% End: