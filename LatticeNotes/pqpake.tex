\newif\iflncs\lncstrue
\documentclass[oribibl,envcountsect,envcountsame]{llncs}
%\usepackage{times}        // Not Allowed for TCC ---Daniel
%\usepackage{fullpage}    // Not Allowed for TCC submission ---Daniel
\usepackage{microtype}
\usepackage{tikz}
\usepackage{amsmath,amsfonts,amssymb}
%\usepackage{hyperref}
% mathabx overrides many math fonts with ugly ones; instead, add
% individual commands in head.tex (see declaration of \boxplus etc.)
%\usepackage{mathabx}
%\usepackage[amsmath,amsthm,thmmarks,hyperref]{ntheorem}

\usepackage{float}
\usepackage{mathtools}
\usepackage{url}
\usepackage{xspace}
\usepackage{enumitem}

% show equation numbers only for referenced equations
%\mathtoolsset{showonlyrefs}

\newif\iffull\fulltrue


\input{head}
%% Macros for this paper

\newcommand{\comptrap}{\algo{CompTrapdoor}}
\newcommand{\tdfencode}{\algo{Encode}}
\newcommand{\evaltd}{\algo{Eval}_{pk}^{td}}
\newcommand{\evalfunc}{\algo{Eval}_{pk}^{func}}
\newcommand{\tdftag}{\algo{Tag}}
\newcommand{\tdfprop}{\algo{Prop}}
\newcommand{\phtdf}{\ensuremath{\text{PHT}}}
\newcommand{\crpunc}{\ensuremath{\text{CRP}}}
\newcommand{\gfqn}{\ensuremath{\text{GF}(q^n)}}
\newcommand{\hatt}{\ensuremath{\hat{t}}}
%\newcommand{\barA}{\ensuremath{\bar{\matA}}


\newcommand{\Lat}{\Lambda}

\newcommand{\setbad}{\textsf{BAD}}
\newcommand{\setguesses}{\textsf{GUESSES}}


\usepackage{xcolor}
\newcommand{\daniel}[1]{Daniel: #1}
\newcommand{\todo}[1]{TO-DO: #1}

\pagestyle{plain}


%%%%

\title{Thoughts on PQ-PAKE}

\begin{document}
\section{Basic Thoughts}

Let $ 1 < k \in \Z$. Let $R$ be the ring of integers of some cyclotomic field (say 512 for Kyber compatibility).

In this security game, we receive from $\Adv$
 some $\matA \in R_q^{k \times k}, \vecb \in R_q^{k}$, both of which can potentially
be any value. 

We sample uniformly random $\matA_0, \matA_1 \in R_q{k \times k}$, 
gaussians $\vecr \gets \chi^{k}$, $\vece \gets \chi^{k}$, $\hat{e} \gets \chi$,
$\beta \gets \bit$

and send back 

$$\matA_0, \matA_1, \vecu := \matA_{\beta}\vecr + \vece, \vecv := \vecb^{t}\vecr + \hat{e}.$$

It is necessary for security that a malicious adversary will have negligible advantage in
distinguishing $\beta=0$ from $\beta=1$.


I \textit{believe} that requiring $\matA$ to be invertible won't kill the protocol? 


We should be able to do a straightforward if not too efficient
porting of the extended-LWE internals found in Brakerski et al's "Classical Hardness of Learning with Errors"
paper~\cite{DBLP:conf/stoc/BrakerskiLPRS13}.  

\section{Extended LWE}
\label{sec:ext-lwe}

Recall from the formulation of the problem as in Brakerski et al's paper~\cite{DBLP:conf/stoc/BrakerskiLPRS13} 
that the extended-LWE assumption (reformulated for arbitrary dimension module lattices 
over arbitrary base rings) is as follows. 

\begin{definition}
Let $\calZ \subseteq R^{k}$. 
The adversary $\Adv$ can choose an arbitrary $\vecz \in \calZ$, and sends it to the challenger.

The challenger returns 
$$(\matA, \vecb, \inner{\vece, \vecz}+\hat{e})$$

and the adversary must distinguish between two cases. In the first case $\matA$ is chosen uniformly 
at random, $\vecs, \vece \gets \chi^{k}$, $\hat{e} \gets \zeta$ (which may be the same as $\chi$ or may be uniquely 0)
 and $\vecb := \matA^{t}\vecs + \vece$. 

In the second case, everything besides $\vecb$ is chosen in the same way, but $\vecb$ is chosen uniformly at random 
and independently of everything else.
\end{definition}









\section{Reducing Security of Protocol to Hardness of Extended LWE}
\label{sec:proof-ext-lw}

We can use an $\Adv$ breaking the PQ-PAKE described above to attack extended LWE as follows. 


\begin{proof}We receive from the PQ-PAKE adversary $(\tilde{\matA} \in R_q^{k \times k}, \tilde{\vecb} \in R_q^{k}$). 


Let $\vecz = \tilde{\vecb}$, and send $\vecz$ to the extended-LWE challenger, 
receiving back $\matA \in R_q^{k \times k}, \vecb, y := \inner{\vece, \vecz}+\hat{e}$. 

Abort if $\matA$ is not invertible (with non-negligible probability it should be invertible).


Otherwise, test the adversaries behavior for both $\beta=0$ and $\beta=1$ upon setting
$\matA_{\beta} = \matA^{-1}$, $\vecu := \matA^{-t}\vecb$, $v := y$,
$\matA_{1-\beta} \gets R_q^{k \times k}$ uniformly at random.


If the extended-LWE instance was the first case above, then we have that 

$\vecu := \matA^{-t}\vece+\vecs = \matA_{\beta}^{t}\vece+\vecs$ and 
$v := \tilde{\vecb}^t\vece + \hat{e}$, so we have perfectly simulated the PQ-PAKE security game, 
and $\Adv$ will have a non-negligible advantage in distinguishing $\beta=0$ from $\beta=1$. 


However, if the extended-LWE instance was the uniform case above (where the returned $\vecb$ is 
chosen uniformly and independently), then, since $\matA$ is invertible and $\vecb$ is
chosen uniformly at random and independently, $\vecu$ will be statistically 
indistinguishable from uniform over the view of $\Adv$, meaning that nothing whatsoever about 
$\beta$ is leaked to the adversary and so the adversary will have 0 advantage in 
distinguishing $\beta=0$ from $\beta=1$. 
\end{proof}

It remains to show that this formulation of extended-LWE can be reasonably seen as hard. 

\section{Reducing Extended-(M)-LWE to (Lower Dimension) (M)-LWE}
\label{sec:red-ext-lwe-lwe}
Using similar arguments as in the Classical Hardness of Learning with Errors Paper, we should 
be able to show that breaking $\textsf{Ext-LWE}_{R,k,q}$ is no easier than breaking 
$\textsf{LWE}_{R,k-1,q}$ (with slightly less noise).

\subsection{First Is Errorless (M)-LWE}
\label{sec:first-errorless-m-lwe}

We refer to the paper above for the definition of this problem, namely the first ring element of 
$\vecb$ is errorless, and we show that $\textsf{1st Errorless LWE}_{R,k,q,\alpha}$ is no easier than breaking 
$\textsf{LWE}_{R,k-1,q,\alpha}$.

\begin{lemma}
There is an efficient transformation/reduction from $\mlwe_{R,k-1,q,\alpha}$ with \textit{uniform 
secrets} to $\textsf{1st Errorless MLWE}_{R,k-1,q,\alpha}$ with \textit{uniform secrets}. 
\end{lemma}

\begin{proof}
We begin with access to an $\lwe_{R,k-1,q,\alpha}$ oracle. 
For simplicity, we assume $q \in \Z$ is prime, the proof can be adapted if necessary. 

First, we sample $\veca' \gets R_q^{k}$ uniformly at random. We then find an $\hat{\veca} \in R_q^{k}$ 
such that 
\begin{enumerate}
\item The coordinates of $\hat{\veca}$ generate all of $R$, namely that
$\sum_{i \in [k]}^{\inner{\hat{a}_i}} = R$ where $\inner{hat{a}_i}$ is the principal ideal generated by 
$a_i$. 
\item There exists $\kappa \in R_q$  such that $\veca' = \kappa\hat{\veca}$. 
\end{enumerate}

We can efficiently find such a $\hat{\veca}$ and $\kappa$ by using the CRT (Chinese Remainder Theorem) 
decomposition (see, e.g.~\cite{cryptoeprint:2011:133} for the necessary lattice cryptography-oriented background).

Since the number theoretic transform to switch from coefficient representation to CRT representation
can be computed extremely efficiently~\cite{longa2016speeding}, we may assume that each element of 
$\veca'$ is already in CRT representation. 

Concretely (assuming we have sampled ), we compute the number theoretic transform of every element 

Let $S$ be the set of coefficient positions (in the CRT representation) $j$ such 
$a'_{ij} = 0$ for all $i \in [k]$.

Then we may set (note that $\hat{a}_{ij}$ means the jth coefficient in the CRT representation of the $i$th element of $\hat{\veca}$) 
\[
\hat{a}_{ij} = \begin{cases}a_{ij}'&\quad \text{ if } j \notin s\\
                    1&\quad \text{ if} j \in S\end{cases} 
\]
and 
\[\kappa_{j} = \begin{cases}1&\quad\text{ if } j \notin S\\
                            0&\quad\text{ otherwise}\end{cases}\]

The remainder of the proof mostly follows~\cite{DBLP:conf/stoc/BrakerskiLPRS13}. We create a 
matrix $U \in R_q^{k \times k}$ (invertible over $R_q$) such that its leftmost column is $\hat{\veca}$. 
Such as matrix exists, and can be found efficiently. 
\jnote{If there's no obvious faster way, we can choose the rest of the columns
 at random and show it has a non-negligible 
chance of being invertible}




\end{proof}

~\cite{DBLP:conf/crypto/ApplebaumCPS09}
The rest of the proof for first-is-errorless LWE should be a straightforward mapping of the 
Classical Hardness paper.

\subsection{Extended-LWE}

I don't really feel like writing this last step out in module form before we've had a chance to discuss
but it seems like it should work as well. 








\bibliographystyle{alpha}
\bibliography{lattices,crypto,ibe,fhe,numtheory}

\end{document}


%%% Local Variables:
%%% mode: latex
%%% TeX-master: t
%%% End:
