\newif\iflncs\lncstrue
\documentclass[oribibl,envcountsect,envcountsame]{llncs}
%\usepackage{times}        // Not Allowed for TCC ---Daniel
%\usepackage{fullpage}    // Not Allowed for TCC submission ---Daniel
\usepackage{microtype}
\usepackage{tikz}
\usepackage{amsmath,amsfonts,amssymb}
%\usepackage{hyperref}
% mathabx overrides many math fonts with ugly ones; instead, add
% individual commands in head.tex (see declaration of \boxplus etc.)
%\usepackage{mathabx}
%\usepackage[amsmath,amsthm,thmmarks,hyperref]{ntheorem}

\usepackage{float}
\usepackage{mathtools}
\usepackage{url}
\usepackage{xspace}
\usepackage{enumitem}

% show equation numbers only for referenced equations
%\mathtoolsset{showonlyrefs}

\newif\iffull\fulltrue


\input{head}
%% Macros for this paper

\newcommand{\comptrap}{\algo{CompTrapdoor}}
\newcommand{\tdfencode}{\algo{Encode}}
\newcommand{\evaltd}{\algo{Eval}_{pk}^{td}}
\newcommand{\evalfunc}{\algo{Eval}_{pk}^{func}}
\newcommand{\tdftag}{\algo{Tag}}
\newcommand{\tdfprop}{\algo{Prop}}
\newcommand{\phtdf}{\ensuremath{\text{PHT}}}
\newcommand{\crpunc}{\ensuremath{\text{CRP}}}
\newcommand{\gfqn}{\ensuremath{\text{GF}(q^n)}}
\newcommand{\hatt}{\ensuremath{\hat{t}}}
%\newcommand{\barA}{\ensuremath{\bar{\matA}}


\newcommand{\Lat}{\Lambda}

\newcommand{\setbad}{\textsf{BAD}}
\newcommand{\setguesses}{\textsf{GUESSES}}


\usepackage{xcolor}
\newcommand{\daniel}[1]{Daniel: #1}
\newcommand{\todo}[1]{TO-DO: #1}

\pagestyle{plain}


%%%%

\title{Thoughts on PQ-PAKE}

\begin{document}
\section{Basic Thoughts}

Let $1 \leq \ell < k$. Let $R$ be the ring of integers of some cyclotomic field (say 512 for Kyber compatibility).

In this security game, we receive from $\Adv$
 some $\matA \in R_q^{\ell \times k}, \vecb \in R_q^{k}$, both of which can potentially
be any value. 

We sample uniformly random $\matA_0, \matA_1 \in R_q{\ell \times k}$, 
gaussians $\vecr \gets \chi^{k}$, $\vece \gets \chi^{\ell}$, $\hat{e} \gets \chi$,
$\beta \gets bit$

and send back 

$$\matA_0, \matA_1, \vecu := \matA_{\beta}\vecr + \vece, \vecv := \vecb^{t}\vecr + \hat{e}.$$

It is necessary for security that a malicious adversary will have negligible advantage in
distinguishing $\beta=0$ from $\beta=1$.

To prove this, we can do a straightforward porting of ~\cite{DBLP:conf/stoc/BrakerskiLPRS13}





\bibliographystyle{alpha}
\bibliography{lattices,crypto,ibe,fhe,numtheory}

\end{document}


%%% Local Variables:
%%% mode: latex
%%% TeX-master: t
%%% End:
